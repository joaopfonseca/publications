\documentclass[parskip=full]{scrartcl}

\pdfoutput=1

\title{Geometric SMOTENC \\ \LARGE{A geometrically enhanced drop-in
replacement for SMOTENC}}

\author{%
	Joao Fonseca\(^{1*}\), Georgios Douzas\(^{1}\), Fernando Bacao\(^{1}\)
	\\
	\small{\(^{1}\)NOVA Information Management School, Universidade Nova de Lisboa}
	\\
	\small{*Corresponding Author}
	\\
	\\
	\small{Postal Address: NOVA Information Management School, Campus de
    Campolide, 1070--312 Lisboa, Portugal}
	\\
	\small{Telephone: +351 21 382 8610}
}

\usepackage{breakcites}
\usepackage{float}
\usepackage{graphicx}
\usepackage{geometry}
\geometry{%
	a4paper,
	left=18mm,
	right=18mm,
	top=8mm,
}
\usepackage{amsmath}
\usepackage{enumitem}
\usepackage[ruled,vlined]{algorithm2e}
\usepackage{booktabs}
\usepackage{pgfplotstable}
\pgfplotsset{compat=1.14}
\usepackage{longtable}
\usepackage{tabu}
\usepackage{hyperref}
\date{}

\begin{document}

\maketitle

\begin{abstract}
    This is an abstract.
\end{abstract}

\section{Introduction}

This is text~\cite{Chawla2002}.

%\begin{figure}[H]
%	\centering
%	\includegraphics[width=12cm, keepaspectratio]{../analysis/fig3}
%    \captionbelow{An instance belonging to a minority class cluster and one of
%    its 5-nearest neighbors are selected. An observation belonging to the same
%    cluster is generated.}
%\end{figure}

\begin{longtable}{cccccccc}
\caption{Description of the datasets collected after data preprocessing. The sampling strategy is similar across datasets. Legend: (IR) Imbalance Ratio}
\label{tbl:datasets_description}\\
\toprule
           Dataset &  Metric &  Non-Metric &  Obs. &  Min. Obs. &  Maj. Obs. &     IR &  Classes \\
\midrule
\endfirsthead
\caption[]{Description of the datasets collected after data preprocessing. The sampling strategy is similar across datasets. Legend: (IR) Imbalance Ratio} \\
\toprule
           Dataset &  Metric &  Non-Metric &  Obs. &  Min. Obs. &  Maj. Obs. &     IR &  Classes \\
\midrule
\endhead
\midrule
\multicolumn{8}{r}{{Continued on next page}} \\
\midrule
\endfoot

\bottomrule
\endlastfoot
           Abalone &       7 &           1 &  4139 &         15 &        689 &  45.93 &       18 \\
             Adult &       6 &           8 &  5000 &       1268 &       3732 &   2.94 &        2 \\
        Adult (10) &       6 &           8 &  5000 &        451 &       4549 &  10.09 &        2 \\
         Annealing &       6 &           4 &   790 &         34 &        608 &  17.88 &        4 \\
            Census &       7 &          24 &  5000 &        337 &       4663 &  13.84 &        2 \\
     Contraceptive &       5 &           4 &  1473 &        333 &        629 &   1.89 &        3 \\
Contraceptive (10) &       5 &           4 &  1036 &         62 &        629 &  10.15 &        3 \\
Contraceptive (20) &       5 &           4 &   990 &         31 &        629 &  20.29 &        3 \\
Contraceptive (31) &       5 &           4 &   973 &         20 &        629 &  31.45 &        3 \\
Contraceptive (41) &       5 &           4 &   966 &         15 &        629 &  41.93 &        3 \\
         Covertype &      10 &           2 &  5000 &         20 &       2449 & 122.45 &        7 \\
   Credit Approval &       6 &           9 &   653 &        296 &        357 &   1.21 &        2 \\
     German Credit &       7 &          13 &  1000 &        300 &        700 &   2.33 &        2 \\
German Credit (10) &       7 &          13 &   770 &         70 &        700 &  10.00 &        2 \\
German Credit (20) &       7 &          13 &   735 &         35 &        700 &  20.00 &        2 \\
German Credit (30) &       7 &          13 &   723 &         23 &        700 &  30.43 &        2 \\
German Credit (41) &       7 &          13 &   717 &         17 &        700 &  41.18 &        2 \\
     Heart Disease &       5 &           5 &   740 &         22 &        357 &  16.23 &        5 \\
Heart Disease (21) &       5 &           5 &   735 &         17 &        357 &  21.00 &        5 \\
\end{longtable}


\begin{longtable}{ccccccc}
\caption{Mean rankings over the different datasets, folds and runs used in the experiment.}
\label{tbl:mean_sem_ranks}\\
\toprule
Classifier &  Metric &                G-SMOTENC &                     NONE &         SMOTENC &                      ROS &             RUS \\
\midrule
\endfirsthead
\caption[]{Mean rankings over the different datasets, folds and runs used in the experiment.} \\
\toprule
Classifier &  Metric &                G-SMOTENC &                     NONE &         SMOTENC &                      ROS &             RUS \\
\midrule
\endhead
\midrule
\multicolumn{7}{r}{{Continued on next page}} \\
\midrule
\endfoot

\bottomrule
\endlastfoot
        DT &      OA &          1.66 $\pm$ 0.13 & \textbf{1.55 $\pm$ 0.22} & 3.16 $\pm$ 0.16 &          4.00 $\pm$ 0.08 & 4.63 $\pm$ 0.19 \\
        DT & F-Score & \textbf{1.11 $\pm$ 0.07} &          3.21 $\pm$ 0.30 & 2.58 $\pm$ 0.18 &          3.53 $\pm$ 0.16 & 4.58 $\pm$ 0.19 \\
        DT &  G-Mean & \textbf{1.53 $\pm$ 0.21} &          4.89 $\pm$ 0.07 & 2.53 $\pm$ 0.18 &          2.47 $\pm$ 0.23 & 3.58 $\pm$ 0.23 \\
       KNN &      OA &          2.39 $\pm$ 0.12 & \textbf{1.32 $\pm$ 0.23} & 3.58 $\pm$ 0.16 &          2.97 $\pm$ 0.26 & 4.74 $\pm$ 0.17 \\
       KNN & F-Score & \textbf{1.37 $\pm$ 0.16} &          3.37 $\pm$ 0.28 & 2.68 $\pm$ 0.20 &          2.95 $\pm$ 0.27 & 4.63 $\pm$ 0.17 \\
       KNN &  G-Mean & \textbf{1.74 $\pm$ 0.17} &          4.84 $\pm$ 0.12 & 2.63 $\pm$ 0.17 &          3.26 $\pm$ 0.25 & 2.53 $\pm$ 0.35 \\
        LR &      OA &          2.47 $\pm$ 0.15 & \textbf{1.32 $\pm$ 0.23} & 2.76 $\pm$ 0.17 &          3.66 $\pm$ 0.21 & 4.79 $\pm$ 0.16 \\
        LR & F-Score & \textbf{1.89 $\pm$ 0.21} &          3.84 $\pm$ 0.28 & 2.05 $\pm$ 0.24 &          2.79 $\pm$ 0.25 & 4.42 $\pm$ 0.21 \\
        LR &  G-Mean &          1.97 $\pm$ 0.23 &          5.00 $\pm$ 0.00 & 3.29 $\pm$ 0.17 & \textbf{1.89 $\pm$ 0.17} & 2.84 $\pm$ 0.30 \\
        RF &      OA &          1.76 $\pm$ 0.09 & \textbf{1.24 $\pm$ 0.09} & 3.37 $\pm$ 0.11 &          3.66 $\pm$ 0.12 & 4.97 $\pm$ 0.03 \\
        RF & F-Score & \textbf{1.26 $\pm$ 0.13} &          4.21 $\pm$ 0.25 & 2.68 $\pm$ 0.17 &          2.42 $\pm$ 0.22 & 4.42 $\pm$ 0.12 \\
        RF &  G-Mean & \textbf{1.68 $\pm$ 0.22} &          4.84 $\pm$ 0.16 & 2.89 $\pm$ 0.21 &          2.26 $\pm$ 0.23 & 3.32 $\pm$ 0.25 \\
\end{longtable}


\bibliography{references}
\bibliographystyle{ieeetr}

\appendix

\section{Appendix}

\begin{longtable}{cccccccc}
\caption{Wide optimal results}
\label{tbl:wide_optimal}\\
\toprule
           Dataset & Classifier &  Metric &      G-SMOTENC &           NONE &        SMOTENC &            ROS &            RUS \\
\midrule
\endfirsthead
\caption[]{Wide optimal results} \\
\toprule
           Dataset & Classifier &  Metric &      G-SMOTENC &           NONE &        SMOTENC &            ROS &            RUS \\
\midrule
\endhead
\midrule
\multicolumn{8}{r}{{Continued on next page}} \\
\midrule
\endfoot

\bottomrule
\endlastfoot
           Abalone &         DT &      OA &          0.221 & \textbf{0.256} &          0.190 &          0.203 &          0.207 \\
           Abalone &         DT & F-Score &          0.168 & \textbf{0.170} &          0.156 &          0.154 &          0.132 \\
           Abalone &         DT &  G-Mean & \textbf{0.460} &          0.413 &          0.445 &          0.457 &          0.421 \\
           Abalone &        KNN &      OA &          0.215 & \textbf{0.237} &          0.186 &          0.197 &          0.188 \\
           Abalone &        KNN & F-Score & \textbf{0.167} &          0.157 &          0.150 &          0.151 &          0.140 \\
           Abalone &        KNN &  G-Mean & \textbf{0.429} &          0.391 &          0.409 &          0.397 &          0.421 \\
           Abalone &         LR &      OA &          0.235 & \textbf{0.272} &          0.228 &          0.229 &          0.195 \\
           Abalone &         LR & F-Score & \textbf{0.189} &          0.180 &          0.186 &          0.179 &          0.166 \\
           Abalone &         LR &  G-Mean & \textbf{0.473} &          0.415 &          0.466 &          0.456 &          0.441 \\
           Abalone &         RF &      OA &          0.237 & \textbf{0.276} &          0.221 &          0.224 &          0.197 \\
           Abalone &         RF & F-Score & \textbf{0.194} &          0.174 &          0.180 &          0.184 &          0.162 \\
           Abalone &         RF &  G-Mean & \textbf{0.486} &          0.416 &          0.461 &          0.465 &          0.448 \\
             Adult &         DT &      OA &          0.830 & \textbf{0.835} &          0.785 &          0.800 &          0.785 \\
             Adult &         DT & F-Score & \textbf{0.767} &          0.763 &          0.754 &          0.755 &          0.744 \\
             Adult &         DT &  G-Mean & \textbf{0.809} &          0.747 &          0.808 &          0.806 &          0.801 \\
             Adult &        KNN &      OA &          0.786 & \textbf{0.805} &          0.781 &          0.763 &          0.761 \\
             Adult &        KNN & F-Score & \textbf{0.738} &          0.732 &          0.735 &          0.718 &          0.728 \\
             Adult &        KNN &  G-Mean &          0.766 &          0.724 &          0.762 &          0.757 & \textbf{0.780} \\
             Adult &         LR &      OA &          0.803 & \textbf{0.839} &          0.803 &          0.804 &          0.801 \\
             Adult &         LR & F-Score &          0.768 & \textbf{0.773} &          0.767 &          0.771 &          0.769 \\
             Adult &         LR &  G-Mean &          0.813 &          0.758 &          0.805 & \textbf{0.815} & \textbf{0.815} \\
             Adult &         RF &      OA &          0.820 & \textbf{0.832} &          0.757 &          0.755 &          0.753 \\
             Adult &         RF & F-Score & \textbf{0.769} &          0.739 &          0.727 &          0.729 &          0.728 \\
             Adult &         RF &  G-Mean &          0.796 &          0.711 &          0.787 & \textbf{0.797} & \textbf{0.797} \\
        Adult (10) &         DT &      OA & \textbf{0.930} &          0.928 &          0.822 &          0.789 &          0.775 \\
        Adult (10) &         DT & F-Score & \textbf{0.711} &          0.708 &          0.656 &          0.641 &          0.630 \\
        Adult (10) &         DT &  G-Mean &          0.812 &          0.663 &          0.807 & \textbf{0.815} &          0.808 \\
        Adult (10) &        KNN &      OA &          0.864 & \textbf{0.909} &          0.854 &          0.851 &          0.745 \\
        Adult (10) &        KNN & F-Score & \textbf{0.667} &          0.652 &          0.658 &          0.648 &          0.602 \\
        Adult (10) &        KNN &  G-Mean &          0.745 &          0.629 &          0.747 &          0.722 & \textbf{0.783} \\
        Adult (10) &         LR &      OA &          0.836 & \textbf{0.925} &          0.837 &          0.815 &          0.791 \\
        Adult (10) &         LR & F-Score &          0.666 & \textbf{0.705} &          0.667 &          0.663 &          0.647 \\
        Adult (10) &         LR &  G-Mean &          0.804 &          0.663 &          0.787 &          0.811 & \textbf{0.814} \\
        Adult (10) &         RF &      OA &          0.899 & \textbf{0.924} &          0.773 &          0.763 &          0.743 \\
        Adult (10) &         RF & F-Score & \textbf{0.718} &          0.615 &          0.620 &          0.624 &          0.610 \\
        Adult (10) &         RF &  G-Mean & \textbf{0.809} &          0.579 &          0.786 &          0.806 &          0.806 \\
         Annealing &         DT &      OA &          0.824 & \textbf{0.843} &          0.742 &          0.733 &          0.694 \\
         Annealing &         DT & F-Score & \textbf{0.736} &          0.643 &          0.732 &          0.724 &          0.683 \\
         Annealing &         DT &  G-Mean & \textbf{0.914} &          0.738 &          0.909 &          0.906 &          0.880 \\
         Annealing &        KNN &      OA &          0.849 &          0.847 &          0.829 & \textbf{0.854} &          0.508 \\
         Annealing &        KNN & F-Score &          0.780 &          0.724 &          0.747 & \textbf{0.783} &          0.476 \\
         Annealing &        KNN &  G-Mean &          0.901 &          0.781 &          0.867 & \textbf{0.909} &          0.814 \\
         Annealing &         LR &      OA &          0.572 & \textbf{0.814} &          0.573 &          0.566 &          0.510 \\
         Annealing &         LR & F-Score & \textbf{0.620} &          0.540 &          0.617 &          0.615 &          0.496 \\
         Annealing &         LR &  G-Mean & \textbf{0.851} &          0.663 &          0.843 &          0.848 &          0.811 \\
         Annealing &         RF &      OA & \textbf{0.868} & \textbf{0.868} &          0.729 &          0.733 &          0.637 \\
         Annealing &         RF & F-Score & \textbf{0.800} &          0.644 &          0.730 &          0.736 &          0.641 \\
         Annealing &         RF &  G-Mean & \textbf{0.917} &          0.727 &          0.904 &          0.910 &          0.873 \\
            Census &         DT &      OA &          0.942 & \textbf{0.943} &          0.894 &          0.844 &          0.795 \\
            Census &         DT & F-Score & \textbf{0.733} &          0.731 &          0.693 &          0.652 &          0.617 \\
            Census &         DT &  G-Mean &          0.813 &          0.698 &          0.800 &          0.814 & \textbf{0.817} \\
            Census &        KNN &      OA &          0.874 & \textbf{0.933} &          0.867 &          0.878 &          0.731 \\
            Census &        KNN & F-Score &          0.652 &          0.648 & \textbf{0.655} &          0.640 &          0.567 \\
            Census &        KNN &  G-Mean &          0.767 &          0.620 &          0.768 &          0.733 & \textbf{0.794} \\
            Census &         LR &      OA &          0.940 & \textbf{0.949} &          0.938 &          0.940 &          0.815 \\
            Census &         LR & F-Score &          0.760 &          0.743 &          0.760 & \textbf{0.762} &          0.639 \\
            Census &         LR &  G-Mean &          0.807 &          0.707 &          0.782 &          0.801 & \textbf{0.837} \\
            Census &         RF &      OA &          0.876 & \textbf{0.933} &          0.819 &          0.740 &          0.714 \\
            Census &         RF & F-Score & \textbf{0.679} &          0.483 &          0.636 &          0.580 &          0.562 \\
            Census &         RF &  G-Mean & \textbf{0.827} &          0.500 &          0.818 &          0.822 &          0.814 \\
     Contraceptive &         DT &      OA & \textbf{0.563} &          0.538 &          0.537 &          0.512 &          0.525 \\
     Contraceptive &         DT & F-Score & \textbf{0.549} &          0.518 &          0.529 &          0.507 &          0.520 \\
     Contraceptive &         DT &  G-Mean & \textbf{0.661} &          0.630 &          0.646 &          0.630 &          0.641 \\
     Contraceptive &        KNN &      OA &          0.465 & \textbf{0.478} &          0.455 &          0.435 &          0.468 \\
     Contraceptive &        KNN & F-Score &          0.460 & \textbf{0.462} &          0.450 &          0.432 &          0.461 \\
     Contraceptive &        KNN &  G-Mean &          0.588 &          0.580 &          0.579 &          0.566 & \textbf{0.590} \\
     Contraceptive &         LR &      OA & \textbf{0.515} &          0.514 &          0.514 &          0.510 &          0.510 \\
     Contraceptive &         LR & F-Score & \textbf{0.512} &          0.492 &          0.509 &          0.505 &          0.506 \\
     Contraceptive &         LR &  G-Mean & \textbf{0.635} &          0.604 &          0.631 &          0.628 &          0.627 \\
     Contraceptive &         RF &      OA &          0.553 & \textbf{0.557} &          0.540 &          0.534 &          0.526 \\
     Contraceptive &         RF & F-Score & \textbf{0.545} &          0.524 &          0.535 &          0.529 &          0.522 \\
     Contraceptive &         RF &  G-Mean & \textbf{0.659} &          0.634 &          0.653 &          0.649 &          0.643 \\
Contraceptive (10) &         DT &      OA & \textbf{0.645} & \textbf{0.645} &          0.568 &          0.528 &          0.487 \\
Contraceptive (10) &         DT & F-Score & \textbf{0.479} &          0.452 &          0.478 &          0.454 &          0.414 \\
Contraceptive (10) &         DT &  G-Mean &          0.644 &          0.584 & \textbf{0.648} &          0.637 &          0.610 \\
Contraceptive (10) &        KNN &      OA &          0.524 & \textbf{0.570} &          0.508 &          0.495 &          0.451 \\
Contraceptive (10) &        KNN & F-Score & \textbf{0.419} &          0.404 &          0.410 &          0.404 &          0.368 \\
Contraceptive (10) &        KNN &  G-Mean & \textbf{0.576} &          0.529 &          0.561 &          0.569 &          0.561 \\
Contraceptive (10) &         LR &      OA &          0.516 & \textbf{0.622} &          0.506 &          0.489 &          0.476 \\
Contraceptive (10) &         LR & F-Score & \textbf{0.431} &          0.375 &          0.426 &          0.425 &          0.411 \\
Contraceptive (10) &         LR &  G-Mean &          0.619 &          0.526 &          0.609 & \textbf{0.624} &          0.618 \\
Contraceptive (10) &         RF &      OA &          0.648 & \textbf{0.651} &          0.569 &          0.550 &          0.494 \\
Contraceptive (10) &         RF & F-Score & \textbf{0.500} &          0.387 &          0.473 &          0.471 &          0.425 \\
Contraceptive (10) &         RF &  G-Mean & \textbf{0.656} &          0.542 &          0.639 &          0.650 &          0.625 \\
Contraceptive (20) &         DT &      OA & \textbf{0.671} &          0.659 &          0.612 &          0.556 &          0.456 \\
Contraceptive (20) &         DT & F-Score & \textbf{0.475} &          0.430 &          0.459 &          0.428 &          0.371 \\
Contraceptive (20) &         DT &  G-Mean & \textbf{0.643} &          0.570 &          0.626 &          0.632 &          0.605 \\
Contraceptive (20) &        KNN &      OA &          0.556 & \textbf{0.600} &          0.529 &          0.541 &          0.442 \\
Contraceptive (20) &        KNN & F-Score & \textbf{0.399} &          0.375 &          0.384 &          0.389 &          0.345 \\
Contraceptive (20) &        KNN &  G-Mean & \textbf{0.565} &          0.519 &          0.544 &          0.537 &          0.549 \\
Contraceptive (20) &         LR &      OA &          0.506 & \textbf{0.641} &          0.508 &          0.486 &          0.440 \\
Contraceptive (20) &         LR & F-Score & \textbf{0.397} &          0.375 & \textbf{0.397} &          0.389 &          0.358 \\
Contraceptive (20) &         LR &  G-Mean &          0.608 &          0.523 &          0.604 & \textbf{0.613} &          0.585 \\
Contraceptive (20) &         RF &      OA &          0.668 & \textbf{0.674} &          0.588 &          0.562 &          0.475 \\
Contraceptive (20) &         RF & F-Score & \textbf{0.473} &          0.384 &          0.450 &          0.436 &          0.389 \\
Contraceptive (20) &         RF &  G-Mean &          0.659 &          0.535 &          0.641 & \textbf{0.670} &          0.633 \\
Contraceptive (31) &         DT &      OA &          0.667 & \textbf{0.670} &          0.608 &          0.604 &          0.440 \\
Contraceptive (31) &         DT & F-Score & \textbf{0.454} &          0.441 &          0.438 &          0.453 &          0.346 \\
Contraceptive (31) &         DT &  G-Mean &          0.642 &          0.577 &          0.605 & \textbf{0.655} &          0.592 \\
Contraceptive (31) &        KNN &      OA &          0.563 & \textbf{0.633} &          0.545 &          0.550 &          0.405 \\
Contraceptive (31) &        KNN & F-Score & \textbf{0.403} &          0.385 &          0.384 &          0.378 &          0.298 \\
Contraceptive (31) &        KNN &  G-Mean & \textbf{0.574} &          0.527 &          0.544 &          0.531 &          0.511 \\
Contraceptive (31) &         LR &      OA &          0.500 & \textbf{0.656} &          0.508 &          0.483 &          0.423 \\
Contraceptive (31) &         LR & F-Score & \textbf{0.379} &          0.376 & \textbf{0.379} &          0.374 &          0.336 \\
Contraceptive (31) &         LR &  G-Mean & \textbf{0.597} &          0.523 &          0.579 &          0.585 &          0.580 \\
Contraceptive (31) &         RF &      OA &          0.681 & \textbf{0.683} &          0.608 &          0.583 &          0.442 \\
Contraceptive (31) &         RF & F-Score & \textbf{0.450} &          0.378 &          0.434 &          0.435 &          0.349 \\
Contraceptive (31) &         RF &  G-Mean & \textbf{0.647} &          0.531 &          0.630 &          0.640 &          0.600 \\
Contraceptive (41) &         DT &      OA &          0.651 & \textbf{0.666} &          0.588 &          0.566 &          0.433 \\
Contraceptive (41) &         DT & F-Score & \textbf{0.459} &          0.426 &          0.408 &          0.409 &          0.336 \\
Contraceptive (41) &         DT &  G-Mean & \textbf{0.622} &          0.573 &          0.579 &          0.589 &          0.555 \\
Contraceptive (41) &        KNN &      OA &          0.563 & \textbf{0.611} &          0.546 &          0.538 &          0.395 \\
Contraceptive (41) &        KNN & F-Score & \textbf{0.393} &          0.373 &          0.381 &          0.370 &          0.289 \\
Contraceptive (41) &        KNN &  G-Mean &          0.542 &          0.515 & \textbf{0.550} &          0.526 &          0.515 \\
Contraceptive (41) &         LR &      OA &          0.525 & \textbf{0.658} &          0.524 &          0.504 &          0.435 \\
Contraceptive (41) &         LR & F-Score &          0.389 &          0.375 & \textbf{0.393} &          0.387 &          0.336 \\
Contraceptive (41) &         LR &  G-Mean &          0.606 &          0.520 &          0.604 & \textbf{0.627} &          0.569 \\
Contraceptive (41) &         RF &      OA &          0.665 & \textbf{0.681} &          0.598 &          0.588 &          0.415 \\
Contraceptive (41) &         RF & F-Score & \textbf{0.444} &          0.378 &          0.418 &          0.429 &          0.323 \\
Contraceptive (41) &         RF &  G-Mean &          0.612 &          0.528 & \textbf{0.616} & \textbf{0.616} &          0.566 \\
         Covertype &         DT &      OA &          0.580 & \textbf{0.705} &          0.587 &          0.567 &          0.450 \\
         Covertype &         DT & F-Score &          0.484 & \textbf{0.490} &          0.481 &          0.475 &          0.361 \\
         Covertype &         DT &  G-Mean & \textbf{0.769} &          0.671 &          0.758 &          0.758 &          0.700 \\
         Covertype &        KNN &      OA &          0.690 & \textbf{0.700} &          0.683 &          0.699 &          0.454 \\
         Covertype &        KNN & F-Score &          0.532 &          0.457 &          0.535 & \textbf{0.561} &          0.367 \\
         Covertype &        KNN &  G-Mean &          0.745 &          0.642 &          0.753 & \textbf{0.763} &          0.691 \\
         Covertype &         LR &      OA &          0.637 & \textbf{0.721} &          0.640 &          0.611 &          0.472 \\
         Covertype &         LR & F-Score &          0.516 &          0.507 & \textbf{0.526} &          0.492 &          0.353 \\
         Covertype &         LR &  G-Mean & \textbf{0.792} &          0.678 &          0.786 &          0.790 &          0.697 \\
         Covertype &         RF &      OA &          0.598 & \textbf{0.704} &          0.583 &          0.587 &          0.485 \\
         Covertype &         RF & F-Score &          0.517 &          0.360 &          0.507 & \textbf{0.519} &          0.394 \\
         Covertype &         RF &  G-Mean &          0.800 &          0.572 &          0.799 & \textbf{0.804} &          0.737 \\
   Credit Approval &         DT &      OA & \textbf{0.867} &          0.847 &          0.862 &          0.861 &          0.865 \\
   Credit Approval &         DT & F-Score & \textbf{0.867} &          0.845 &          0.862 &          0.861 &          0.865 \\
   Credit Approval &         DT &  G-Mean & \textbf{0.874} &          0.848 &          0.869 &          0.867 &          0.872 \\
   Credit Approval &        KNN &      OA & \textbf{0.870} &          0.865 &          0.868 & \textbf{0.870} &          0.865 \\
   Credit Approval &        KNN & F-Score & \textbf{0.869} &          0.864 &          0.867 & \textbf{0.869} &          0.864 \\
   Credit Approval &        KNN &  G-Mean & \textbf{0.871} &          0.865 &          0.868 & \textbf{0.871} &          0.866 \\
   Credit Approval &         LR &      OA &          0.873 &          0.868 &          0.871 & \textbf{0.874} &          0.873 \\
   Credit Approval &         LR & F-Score &          0.873 &          0.868 &          0.871 & \textbf{0.874} &          0.873 \\
   Credit Approval &         LR &  G-Mean &          0.877 &          0.873 &          0.877 & \textbf{0.879} &          0.878 \\
   Credit Approval &         RF &      OA &          0.876 & \textbf{0.877} &          0.871 &          0.868 &          0.868 \\
   Credit Approval &         RF & F-Score &          0.876 & \textbf{0.877} &          0.871 &          0.868 &          0.868 \\
   Credit Approval &         RF &  G-Mean & \textbf{0.879} & \textbf{0.879} &          0.876 &          0.872 &          0.873 \\
     German Credit &         DT &      OA &          0.704 & \textbf{0.713} &          0.702 &          0.660 &          0.644 \\
     German Credit &         DT & F-Score & \textbf{0.662} &          0.608 &          0.654 &          0.633 &          0.623 \\
     German Credit &         DT &  G-Mean & \textbf{0.681} &          0.608 &          0.667 &          0.663 &          0.660 \\
     German Credit &        KNN &      OA &          0.681 & \textbf{0.718} &          0.682 &          0.670 &          0.641 \\
     German Credit &        KNN & F-Score & \textbf{0.653} &          0.628 &          0.650 &          0.636 &          0.616 \\
     German Credit &        KNN &  G-Mean & \textbf{0.675} &          0.621 &          0.668 &          0.656 &          0.642 \\
     German Credit &         LR &      OA &          0.727 & \textbf{0.751} &          0.729 &          0.724 &          0.712 \\
     German Credit &         LR & F-Score &          0.695 &          0.681 & \textbf{0.697} & \textbf{0.697} &          0.686 \\
     German Credit &         LR &  G-Mean & \textbf{0.722} &          0.672 &          0.713 &          0.720 &          0.713 \\
     German Credit &         RF &      OA & \textbf{0.760} &          0.741 &          0.739 &          0.737 &          0.700 \\
     German Credit &         RF & F-Score &          0.701 &          0.580 &          0.702 & \textbf{0.709} &          0.680 \\
     German Credit &         RF &  G-Mean &          0.715 &          0.588 &          0.716 & \textbf{0.730} &          0.719 \\
German Credit (10) &         DT &      OA & \textbf{0.909} &          0.906 &          0.804 &          0.713 &          0.696 \\
German Credit (10) &         DT & F-Score & \textbf{0.575} &          0.539 &          0.572 &          0.526 &          0.511 \\
German Credit (10) &         DT &  G-Mean &          0.628 &          0.535 &          0.629 & \textbf{0.644} &          0.631 \\
German Credit (10) &        KNN &      OA &          0.787 & \textbf{0.913} &          0.757 &          0.835 &          0.684 \\
German Credit (10) &        KNN & F-Score &          0.578 & \textbf{0.581} &          0.558 &          0.573 &          0.528 \\
German Credit (10) &        KNN &  G-Mean &          0.662 &          0.559 &          0.643 &          0.588 & \textbf{0.667} \\
German Credit (10) &         LR &      OA &          0.839 & \textbf{0.904} &          0.831 &          0.799 &          0.682 \\
German Credit (10) &         LR & F-Score &          0.619 &          0.596 &          0.610 & \textbf{0.620} &          0.550 \\
German Credit (10) &         LR &  G-Mean &          0.683 &          0.578 &          0.675 &          0.716 & \textbf{0.722} \\
German Credit (10) &         RF &      OA & \textbf{0.910} &          0.909 &          0.865 &          0.877 &          0.696 \\
German Credit (10) &         RF & F-Score &          0.624 &          0.476 &          0.614 & \textbf{0.661} &          0.557 \\
German Credit (10) &         RF &  G-Mean &          0.653 &          0.500 &          0.646 &          0.709 & \textbf{0.729} \\
German Credit (20) &         DT &      OA & \textbf{0.952} & \textbf{0.952} &          0.875 &          0.795 &          0.668 \\
German Credit (20) &         DT & F-Score & \textbf{0.573} &          0.525 &          0.559 &          0.522 &          0.457 \\
German Credit (20) &         DT &  G-Mean &          0.666 &          0.529 &          0.679 & \textbf{0.690} &          0.629 \\
German Credit (20) &        KNN &      OA &          0.856 & \textbf{0.952} &          0.826 &          0.905 &          0.679 \\
German Credit (20) &        KNN & F-Score & \textbf{0.561} &          0.535 &          0.528 &          0.556 &          0.491 \\
German Credit (20) &        KNN &  G-Mean &          0.692 &          0.527 &          0.635 &          0.570 & \textbf{0.709} \\
German Credit (20) &         LR &      OA &          0.913 & \textbf{0.952} &          0.910 &          0.838 &          0.680 \\
German Credit (20) &         LR & F-Score & \textbf{0.596} &          0.534 &          0.593 &          0.553 &          0.473 \\
German Credit (20) &         LR &  G-Mean &          0.651 &          0.531 &          0.627 &          0.661 & \textbf{0.682} \\
German Credit (20) &         RF &      OA & \textbf{0.954} &          0.952 &          0.920 &          0.931 &          0.709 \\
German Credit (20) &         RF & F-Score & \textbf{0.597} &          0.488 &          0.574 &          0.572 &          0.493 \\
German Credit (20) &         RF &  G-Mean &          0.681 &          0.500 &          0.625 &          0.674 & \textbf{0.691} \\
German Credit (30) &         DT &      OA & \textbf{0.968} &          0.963 &          0.885 &          0.856 &          0.628 \\
German Credit (30) &         DT & F-Score & \textbf{0.558} &          0.509 &          0.526 &          0.506 &          0.413 \\
German Credit (30) &         DT &  G-Mean & \textbf{0.686} &          0.509 &          0.631 &          0.602 &          0.565 \\
German Credit (30) &        KNN &      OA &          0.902 & \textbf{0.968} &          0.849 &          0.935 &          0.697 \\
German Credit (30) &        KNN & F-Score & \textbf{0.530} &          0.492 &          0.512 &          0.519 &          0.473 \\
German Credit (30) &        KNN &  G-Mean &          0.681 &          0.500 &          0.588 &          0.536 & \textbf{0.705} \\
German Credit (30) &         LR &      OA &          0.921 & \textbf{0.967} &          0.918 &          0.877 &          0.611 \\
German Credit (30) &         LR & F-Score & \textbf{0.578} &          0.516 &          0.577 &          0.537 &          0.421 \\
German Credit (30) &         LR &  G-Mean &          0.649 &          0.510 &          0.650 & \textbf{0.661} &          0.660 \\
German Credit (30) &         RF &      OA & \textbf{0.968} & \textbf{0.968} &          0.942 &          0.954 &          0.705 \\
German Credit (30) &         RF & F-Score & \textbf{0.592} &          0.492 &          0.563 &          0.589 &          0.474 \\
German Credit (30) &         RF &  G-Mean & \textbf{0.689} &          0.500 &          0.601 &          0.606 &          0.679 \\
German Credit (41) &         DT &      OA & \textbf{0.976} &          0.971 &          0.916 &          0.905 &          0.635 \\
German Credit (41) &         DT & F-Score & \textbf{0.563} &          0.493 &          0.544 &          0.502 &          0.408 \\
German Credit (41) &         DT &  G-Mean & \textbf{0.636} &          0.497 &          0.615 &          0.520 &          0.524 \\
German Credit (41) &        KNN &      OA &          0.929 & \textbf{0.976} &          0.876 &          0.944 &          0.674 \\
German Credit (41) &        KNN & F-Score & \textbf{0.524} &          0.494 &          0.500 &          0.502 &          0.440 \\
German Credit (41) &        KNN &  G-Mean &          0.593 &          0.500 &          0.558 &          0.516 & \textbf{0.630} \\
German Credit (41) &         LR &      OA &          0.940 & \textbf{0.976} &          0.943 &          0.927 &          0.641 \\
German Credit (41) &         LR & F-Score &          0.546 &          0.494 & \textbf{0.552} &          0.515 &          0.420 \\
German Credit (41) &         LR &  G-Mean & \textbf{0.602} &          0.500 &          0.592 &          0.598 &          0.597 \\
German Credit (41) &         RF &      OA & \textbf{0.976} & \textbf{0.976} &          0.961 &          0.969 &          0.636 \\
German Credit (41) &         RF & F-Score & \textbf{0.598} &          0.494 &          0.566 &          0.591 &          0.413 \\
German Credit (41) &         RF &  G-Mean &          0.621 &          0.500 & \textbf{0.622} &          0.614 &          0.572 \\
     Heart Disease &         DT &      OA &          0.532 & \textbf{0.566} &          0.509 &          0.473 &          0.430 \\
     Heart Disease &         DT & F-Score & \textbf{0.371} &          0.322 &          0.342 &          0.331 &          0.295 \\
     Heart Disease &         DT &  G-Mean & \textbf{0.588} &          0.534 &          0.563 &          0.545 &          0.515 \\
     Heart Disease &        KNN &      OA &          0.538 & \textbf{0.564} &          0.535 &          0.534 &          0.504 \\
     Heart Disease &        KNN & F-Score & \textbf{0.363} &          0.287 &          0.360 &          0.352 &          0.341 \\
     Heart Disease &        KNN &  G-Mean & \textbf{0.571} &          0.509 & \textbf{0.571} &          0.560 &          0.557 \\
     Heart Disease &         LR &      OA &          0.558 & \textbf{0.584} &          0.557 &          0.536 &          0.480 \\
     Heart Disease &         LR & F-Score & \textbf{0.397} &          0.329 &          0.395 &          0.374 &          0.333 \\
     Heart Disease &         LR &  G-Mean &          0.601 &          0.539 &          0.601 & \textbf{0.603} &          0.567 \\
     Heart Disease &         RF &      OA &          0.553 & \textbf{0.601} &          0.546 &          0.539 &          0.480 \\
     Heart Disease &         RF & F-Score & \textbf{0.385} &          0.314 &          0.366 &          0.360 &          0.326 \\
     Heart Disease &         RF &  G-Mean & \textbf{0.600} &          0.531 &          0.580 &          0.569 &          0.566 \\
Heart Disease (21) &         DT &      OA &          0.532 & \textbf{0.566} &          0.512 &          0.486 &          0.431 \\
Heart Disease (21) &         DT & F-Score & \textbf{0.376} &          0.296 &          0.341 &          0.336 &          0.311 \\
Heart Disease (21) &         DT &  G-Mean & \textbf{0.598} &          0.509 &          0.558 &          0.562 &          0.538 \\
Heart Disease (21) &        KNN &      OA &          0.561 & \textbf{0.569} &          0.543 &          0.541 &          0.491 \\
Heart Disease (21) &        KNN & F-Score & \textbf{0.385} &          0.312 &          0.365 &          0.363 &          0.334 \\
Heart Disease (21) &        KNN &  G-Mean & \textbf{0.589} &          0.520 &          0.570 &          0.566 &          0.546 \\
Heart Disease (21) &         LR &      OA &          0.573 & \textbf{0.592} &          0.565 &          0.547 &          0.525 \\
Heart Disease (21) &         LR & F-Score & \textbf{0.408} &          0.331 &          0.405 &          0.387 &          0.343 \\
Heart Disease (21) &         LR &  G-Mean & \textbf{0.638} &          0.540 &          0.610 &          0.602 &          0.583 \\
Heart Disease (21) &         RF &      OA &          0.577 & \textbf{0.608} &          0.565 &          0.561 &          0.517 \\
Heart Disease (21) &         RF & F-Score & \textbf{0.417} &          0.323 &          0.390 &          0.383 &          0.337 \\
Heart Disease (21) &         RF &  G-Mean & \textbf{0.621} &          0.536 &          0.596 &          0.593 &          0.567 \\
\end{longtable}


\end{document}
