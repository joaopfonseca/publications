\documentclass[parskip=full]{scrartcl}

\pdfoutput=1

\title{Geometric SMOTENC \\ \LARGE{A geometrically enhanced drop-in
replacement for SMOTENC}}

\author{%
	Joao Fonseca\(^{1*}\), Fernando Bacao\(^{1}\)
	\\
	\small{\(^{1}\)NOVA Information Management School, Universidade Nova de Lisboa}
	\\
	\small{*Corresponding Author}
	\\
	\\
	\small{Postal Address: NOVA Information Management School, Campus de
    Campolide, 1070--312 Lisboa, Portugal}
	\\
	\small{Telephone: +351 21 382 8610}
}

\usepackage{breakcites}
\usepackage{float}
\usepackage{graphicx}
\usepackage{geometry}
\geometry{%
	a4paper,
	left=18mm,
	right=18mm,
	top=8mm,
}
\usepackage{amsmath}
\usepackage{enumitem}
\usepackage[ruled,vlined]{algorithm2e}
\usepackage{booktabs}
\usepackage{pgfplotstable}
\pgfplotsset{compat=1.14}
\usepackage{longtable}
\usepackage{tabu}
\usepackage{hyperref}
\date{}

\begin{document}

\maketitle

\begin{abstract}
    This is an abstract.
\end{abstract}

\section{Introduction}~\label{sec:introduction}

This is text~\cite{Chawla2002}.

\section{Related Work}~\label{sec:related_work}

\section{Motivation}~\label{sec:motivation}

\section{Proposed Method}~\label{sec:proposed_method}

\section{Methodology}~\label{sec:methodology}

$C_{maj}$ set of majority class observations (most common class found in the
target variable)

$C_{min}$ set of minority class observations (least common class found in the
target variable)

\subsection{Experimental Data}~\label{sec:experimental_data}

The datasets used in this experiment were extracted from the
\href{https://archive.ics.uci.edu}{UC Irvine Machine Learning Repository}. All
of the datasets are publicly available and cover a range of different domains.
The selection of datasets was done to ensure that all datasets are imbalanced
and contained non-metric features (\textit{i.e.}, whether ordinal, nominal or
binary). These datasets will be used to show how the performance of different
classifiers varies according to the used over/undersampling method.

At an initial stage, all datasets were preprocessed manually with minimal
manipulations, to avoid the application of preprocessing methods beyond the
scope of this paper. This step was conducted to remove features and/or
observations with missing values and identifying the non-metric features. The
second stage of our preprocessing was done systematically. The resulting
datasets are shown in Table~\ref{tbl:datasets_description}.

\begin{longtable}{cccccccc}
\caption{Description of the datasets collected after data preprocessing. The sampling strategy is similar across datasets. Legend: (IR) Imbalance Ratio}
\label{tbl:datasets_description}\\
\toprule
           Dataset &  Metric &  Non-Metric &  Obs. &  Min. Obs. &  Maj. Obs. &     IR &  Classes \\
\midrule
\endfirsthead
\caption[]{Description of the datasets collected after data preprocessing. The sampling strategy is similar across datasets. Legend: (IR) Imbalance Ratio} \\
\toprule
           Dataset &  Metric &  Non-Metric &  Obs. &  Min. Obs. &  Maj. Obs. &     IR &  Classes \\
\midrule
\endhead
\midrule
\multicolumn{8}{r}{{Continued on next page}} \\
\midrule
\endfoot

\bottomrule
\endlastfoot
           Abalone &       7 &           1 &  4139 &         15 &        689 &  45.93 &       18 \\
             Adult &       6 &           8 &  5000 &       1268 &       3732 &   2.94 &        2 \\
        Adult (10) &       6 &           8 &  5000 &        451 &       4549 &  10.09 &        2 \\
         Annealing &       6 &           4 &   790 &         34 &        608 &  17.88 &        4 \\
            Census &       7 &          24 &  5000 &        337 &       4663 &  13.84 &        2 \\
     Contraceptive &       5 &           4 &  1473 &        333 &        629 &   1.89 &        3 \\
Contraceptive (10) &       5 &           4 &  1036 &         62 &        629 &  10.15 &        3 \\
Contraceptive (20) &       5 &           4 &   990 &         31 &        629 &  20.29 &        3 \\
Contraceptive (31) &       5 &           4 &   973 &         20 &        629 &  31.45 &        3 \\
Contraceptive (41) &       5 &           4 &   966 &         15 &        629 &  41.93 &        3 \\
         Covertype &      10 &           2 &  5000 &         20 &       2449 & 122.45 &        7 \\
   Credit Approval &       6 &           9 &   653 &        296 &        357 &   1.21 &        2 \\
     German Credit &       7 &          13 &  1000 &        300 &        700 &   2.33 &        2 \\
German Credit (10) &       7 &          13 &   770 &         70 &        700 &  10.00 &        2 \\
German Credit (20) &       7 &          13 &   735 &         35 &        700 &  20.00 &        2 \\
German Credit (30) &       7 &          13 &   723 &         23 &        700 &  30.43 &        2 \\
German Credit (41) &       7 &          13 &   717 &         17 &        700 &  41.18 &        2 \\
     Heart Disease &       5 &           5 &   740 &         22 &        357 &  16.23 &        5 \\
Heart Disease (21) &       5 &           5 &   735 &         17 &        357 &  21.00 &        5 \\
\end{longtable}


The second part of the data preprocessing pipeline starts with the generation
of artificially imbalanced datasets with different Imbalance Ratios (IR). For
each original dataset, we create its more imbalanced versions at intervals of
10, while ensuring that $|C_{min}| \ge 15$. The sampling strategy was
determined for class $n \in \{1,\ldots,n,\ldots,m\}$ as a linear interpolation using $|C_{maj}|$ and
$|C_{min}'|=\frac{|C_{maj}|}{IR}$, as shown in equation~\ref{eq:sampling}.

\begin{equation}~\label{eq:sampling}
    |C_n|^{imb} =
    \frac{\frac{|C_{maj}|}{IR}-|C_{maj}|}{|\{1,\ldots,n,\ldots,m\}|-1}.|C_n|+|C_{max}|
\end{equation}

The new, artificially imbalanced dataset, is formed by randomly removing
observations from each $C_n$ such that $C_n' \subseteq C_n , |C_n'| =
|C_n|^{imb}$. The artificially imbalanced datasets are marked with its
imbalance ratio as a suffix in Table~\ref{tbl:datasets_description}.

The datasets (both original and artificially imbalanced versions) are then
filtered to ensure all datasets have a minimum of 500 observations.  The
remaining datasets whose number of observations is larger than 5000 are
randomly sampled to match this number of observations. Afterwards, for each
remaining dataset we remove all observations from target classes whose
frequency is lower than 15 observations. Finally, the continuous and discrete
features are scaled to ensure a common range between all features. 

\subsection{Evaluation Measures}~\label{sec:evaluation_measures}

Although the typical performance metrics, \textit{e.g.}, Overall Accuracy
(OA), are intuitive to interpret, they are often inappropriate to measure a
classifier's performance in an imbalanced learning context
\textbf{[CITATION]}. For example, to estimate an event that occurs in 1\% of
the dataset, a constant classifier would obtain an OA of 0.99 and still be
unusable. However, this metric is still reported in some of our results to
maintain a metric that is easier to interpret.

More recent surveys have found the Geometric-mean (G-mean), F1-score
(F-score), sensitivity and specificity to be commonly used performance metrics
in imbalanced learning contexts~\cite{rout2018handling}. This finding is
consistent with other well-known recommendations on the usage of performance
metrics~\cite{jeni2013facing}.  

\subsection{Machine Learning Algorithms}~\label{sec:ml_algorithms}

\subsection{Experimental Procedure}~\label{sec:experimental_procedure}

\subsection{Software Implementation}~\label{sec:software_implementation}

The algorithmic implementation of G-SMOTENC was written using the Python
programming language and is available in the open-source package
\href{https://github.com/joaopfonseca/ml-research}{ML-Research}~\cite{fonseca2021increasing},
along with other utilities used to produce the experiment and outputs used in
Section~\ref{sec:results_and_discussion}. In addition, the packages
\href{https://github.com/scikit-learn/scikit-learn/}{Scikit-Learn}~\cite{scikit-learn},
\href{https://github.com/scikit-learn-contrib/imbalanced-learn}{Imbalanced-Learn}~\cite{JMLR:v18:16-365}
and \href{https://github.com/georgedouzas/research-learn/}{Research-Learn}
were also used in the experimental procedure to get the implementations of the
classifiers, benchmark over/undersamplers and run the experimental procedure.
The Latex code, Python scripts (including data pulling and preprocessing,
experiment setup and results' analysis), as well as the datasets used are
available in this \href{https://github.com/joaopfonseca/publications}{GitHub
repository}.
 

\section{Results and Discussion}~\label{sec:results_and_discussion}

In this section we present the experimental results. We focus on the
comparison of classification performance using oversamplers whose generation
mechanism is compatible with datasets containing both continuous and
categorical features.

%%%%%%%%%%%%%%%%%%%%%%%%%%%%%%%%%%%%%%%%%%%%%%%%%%%%%%%%%%%%%%%%%%%%%%%%%%%%
% Summarize the analysis you did
%%%%%%%%%%%%%%%%%%%%%%%%%%%%%%%%%%%%%%%%%%%%%%%%%%%%%%%%%%%%%%%%%%%%%%%%%%%%

\subsection{Results}~\label{sec:results}

Table~\ref{tbl:mean_sem_ranks} presents the mean rankings of cross validation
scores across the different combinations of oversamplers, metrics and
classifiers. These results were calculated by assigning a ranking score for
each oversampler from 1 (best) to 4 (worst) for each dataset, metric and
classifier, based on the results reported in Table~\ref{tbl:wide_optimal} (See
Appendix).

\begin{longtable}{cccccccc}
\caption{Mean rankings over the different datasets, folds and runs used in the experiment.}
\label{tbl:mean_sem_ranks}\\
\toprule
Classifier &  Metric &                G-SMOTENC &                     NONE &         SMOTENC &                      ROS &             RUS &       SMOTE-ENC \\
\midrule
\endfirsthead
\caption[]{Mean rankings over the different datasets, folds and runs used in the experiment.} \\
\toprule
Classifier &  Metric &                G-SMOTENC &                     NONE &         SMOTENC &                      ROS &             RUS &       SMOTE-ENC \\
\midrule
\endhead
\midrule
\multicolumn{8}{r}{{Continued on next page}} \\
\midrule
\endfoot

\bottomrule
\endlastfoot
        DT &      OA &          1.66 $\pm$ 0.13 & \textbf{1.61 $\pm$ 0.27} & 3.58 $\pm$ 0.20 &          4.68 $\pm$ 0.15 & 5.42 $\pm$ 0.27 & 4.05 $\pm$ 0.23 \\
        DT & F-Score & \textbf{1.32 $\pm$ 0.11} &          3.84 $\pm$ 0.40 & 3.13 $\pm$ 0.20 &          4.32 $\pm$ 0.19 & 5.47 $\pm$ 0.23 & 2.92 $\pm$ 0.34 \\
        DT &  G-Mean & \textbf{1.68 $\pm$ 0.24} &          5.84 $\pm$ 0.09 & 2.82 $\pm$ 0.21 &          2.95 $\pm$ 0.32 & 4.26 $\pm$ 0.32 & 3.45 $\pm$ 0.30 \\
       KNN &      OA &          2.50 $\pm$ 0.17 & \textbf{1.37 $\pm$ 0.28} & 4.21 $\pm$ 0.25 &          3.34 $\pm$ 0.35 & 5.68 $\pm$ 0.22 & 3.89 $\pm$ 0.15 \\
       KNN & F-Score & \textbf{1.37 $\pm$ 0.16} &          3.95 $\pm$ 0.35 & 3.11 $\pm$ 0.29 &          3.47 $\pm$ 0.36 & 5.53 $\pm$ 0.23 & 3.58 $\pm$ 0.23 \\
       KNN &  G-Mean & \textbf{1.74 $\pm$ 0.17} &          5.84 $\pm$ 0.12 & 2.89 $\pm$ 0.23 &          3.76 $\pm$ 0.33 & 3.00 $\pm$ 0.45 & 3.76 $\pm$ 0.23 \\
        LR &      OA &          2.74 $\pm$ 0.19 & \textbf{1.37 $\pm$ 0.28} & 3.08 $\pm$ 0.21 &          4.34 $\pm$ 0.30 & 5.74 $\pm$ 0.17 & 3.74 $\pm$ 0.28 \\
        LR & F-Score & \textbf{2.11 $\pm$ 0.24} &          4.53 $\pm$ 0.35 & 2.37 $\pm$ 0.28 &          3.47 $\pm$ 0.32 & 5.21 $\pm$ 0.27 & 3.32 $\pm$ 0.38 \\
        LR &  G-Mean &          2.13 $\pm$ 0.26 &          6.00 $\pm$ 0.00 & 3.61 $\pm$ 0.21 & \textbf{2.11 $\pm$ 0.23} & 3.32 $\pm$ 0.40 & 3.84 $\pm$ 0.28 \\
        RF &      OA &          1.82 $\pm$ 0.11 & \textbf{1.24 $\pm$ 0.09} & 3.97 $\pm$ 0.16 &          4.32 $\pm$ 0.21 & 5.92 $\pm$ 0.06 & 3.74 $\pm$ 0.22 \\
        RF & F-Score & \textbf{1.32 $\pm$ 0.13} &          5.05 $\pm$ 0.31 & 3.16 $\pm$ 0.22 &          3.05 $\pm$ 0.31 & 5.37 $\pm$ 0.14 & 3.05 $\pm$ 0.27 \\
        RF &  G-Mean & \textbf{1.68 $\pm$ 0.22} &          5.79 $\pm$ 0.21 & 3.26 $\pm$ 0.28 &          2.47 $\pm$ 0.30 & 3.89 $\pm$ 0.35 & 3.89 $\pm$ 0.19 \\
\end{longtable}


Table~\ref{tbl:mean_sem_scores} presents the mean cross validation scores.
With exception to the OA metric, G-SMOTENC either outperformed or matched the
the remaining oversamplers.

\begin{longtable}{cccccccc}
\caption{Mean scores over the different datasets, folds and runs used in the experiment}
\label{tbl:mean_sem_scores}\\
\toprule
Classifier &  Metric &                G-SMOTENC &                     NONE &                  SMOTENC &                      ROS &             RUS &       SMOTE-ENC \\
\midrule
\endfirsthead
\caption[]{Mean scores over the different datasets, folds and runs used in the experiment} \\
\toprule
Classifier &  Metric &                G-SMOTENC &                     NONE &                  SMOTENC &                      ROS &             RUS &       SMOTE-ENC \\
\midrule
\endhead
\midrule
\multicolumn{8}{r}{{Continued on next page}} \\
\midrule
\endfoot

\bottomrule
\endlastfoot
        DT &      OA &          0.74 $\pm$ 0.05 & \textbf{0.75 $\pm$ 0.04} &          0.68 $\pm$ 0.04 &          0.66 $\pm$ 0.04 & 0.58 $\pm$ 0.04 & 0.65 $\pm$ 0.04 \\
        DT & F-Score & \textbf{0.56 $\pm$ 0.04} &          0.52 $\pm$ 0.04 &          0.54 $\pm$ 0.04 &          0.52 $\pm$ 0.04 & 0.48 $\pm$ 0.04 & 0.51 $\pm$ 0.04 \\
        DT &  G-Mean & \textbf{0.69 $\pm$ 0.03} &          0.60 $\pm$ 0.02 &          0.68 $\pm$ 0.03 &          0.67 $\pm$ 0.03 & 0.65 $\pm$ 0.03 & 0.66 $\pm$ 0.03 \\
       KNN &      OA &          0.69 $\pm$ 0.04 & \textbf{0.73 $\pm$ 0.05} &          0.67 $\pm$ 0.04 &          0.69 $\pm$ 0.05 & 0.57 $\pm$ 0.04 & 0.68 $\pm$ 0.05 \\
       KNN & F-Score & \textbf{0.53 $\pm$ 0.04} &          0.50 $\pm$ 0.04 &          0.52 $\pm$ 0.04 &          0.52 $\pm$ 0.04 & 0.46 $\pm$ 0.04 & 0.51 $\pm$ 0.04 \\
       KNN &  G-Mean & \textbf{0.66 $\pm$ 0.03} &          0.58 $\pm$ 0.03 &          0.64 $\pm$ 0.03 &          0.62 $\pm$ 0.03 & 0.65 $\pm$ 0.03 & 0.63 $\pm$ 0.03 \\
        LR &      OA &          0.68 $\pm$ 0.05 & \textbf{0.75 $\pm$ 0.04} &          0.68 $\pm$ 0.05 &          0.66 $\pm$ 0.05 & 0.58 $\pm$ 0.04 & 0.67 $\pm$ 0.04 \\
        LR & F-Score & \textbf{0.54 $\pm$ 0.04} &          0.52 $\pm$ 0.04 & \textbf{0.54 $\pm$ 0.04} &          0.53 $\pm$ 0.04 & 0.48 $\pm$ 0.04 & 0.52 $\pm$ 0.04 \\
        LR &  G-Mean & \textbf{0.69 $\pm$ 0.02} &          0.60 $\pm$ 0.03 &          0.68 $\pm$ 0.02 & \textbf{0.69 $\pm$ 0.03} & 0.67 $\pm$ 0.03 & 0.67 $\pm$ 0.03 \\
        RF &      OA &          0.74 $\pm$ 0.04 & \textbf{0.76 $\pm$ 0.04} &          0.69 $\pm$ 0.04 &          0.69 $\pm$ 0.04 & 0.59 $\pm$ 0.04 & 0.68 $\pm$ 0.05 \\
        RF & F-Score & \textbf{0.57 $\pm$ 0.04} &          0.48 $\pm$ 0.04 &          0.55 $\pm$ 0.04 &          0.55 $\pm$ 0.04 & 0.49 $\pm$ 0.04 & 0.53 $\pm$ 0.04 \\
        RF &  G-Mean & \textbf{0.70 $\pm$ 0.02} &          0.57 $\pm$ 0.02 &          0.68 $\pm$ 0.03 &          0.69 $\pm$ 0.03 & 0.68 $\pm$ 0.03 & 0.68 $\pm$ 0.02 \\
\end{longtable}


\subsection{Statistical Analysis}~\label{sec:statistical_analysis}

The statistical analysis was developed using exclusively imbalance-appropriate
metrics: F-Score and G-Mean. 

\begin{longtable}{cccc}
\caption{Results for the Friedman test. Statistical significance is tested at a level of $\alpha = 0.05$. The null hypothesis is that there is no difference in the classification outcome across resamplers.}
\label{tbl:friedman_test}\\
\toprule
Classifier &  Metric & p-value &  Significance \\
\midrule
\endfirsthead
\caption[]{Results for the Friedman test. Statistical significance is tested at a level of $\alpha = 0.05$. The null hypothesis is that there is no difference in the classification outcome across resamplers.} \\
\toprule
Classifier &  Metric & p-value &  Significance \\
\midrule
\endhead
\midrule
\multicolumn{4}{r}{{Continued on next page}} \\
\midrule
\endfoot

\bottomrule
\endlastfoot
        DT & F-Score & 2.2e-10 &          True \\
        DT &  G-Mean & 1.2e-10 &          True \\
       KNN & F-Score & 2.3e-09 &          True \\
       KNN &  G-Mean & 9.4e-10 &          True \\
        LR & F-Score & 2.1e-07 &          True \\
        LR &  G-Mean & 9.7e-11 &          True \\
        RF & F-Score & 8.5e-12 &          True \\
        RF &  G-Mean & 2.0e-10 &          True \\
\end{longtable}


\begin{longtable}{ccccc}
\caption{Results for Wilcoxon signed-rank method. Statistical significance is tested at a level of $lpha = 0.05$. The null hypothesis is that the performance of the proposed oversampler is similar to the remaining oversamplers.}
\label{tbl:wilcoxon_test}\\
\toprule
           Dataset &     NONE &  RAND-OVER &  RAND-UNDER &  SMOTENC \\
\midrule
\endfirsthead
\caption[]{Results for Wilcoxon signed-rank method. Statistical significance is tested at a level of $lpha = 0.05$. The null hypothesis is that the performance of the proposed oversampler is similar to the remaining oversamplers.} \\
\toprule
           Dataset &     NONE &  RAND-OVER &  RAND-UNDER &  SMOTENC \\
\midrule
\endhead
\midrule
\multicolumn{5}{r}{{Continued on next page}} \\
\midrule
\endfoot

\bottomrule
\endlastfoot
           Abalone & 0.423828 &   0.000488 &    0.000488 & 0.000488 \\
             Adult & 0.266113 &   0.026855 &    0.052246 & 0.003346 \\
        Adult (10) & 0.203613 &   0.006836 &    0.012207 & 0.006836 \\
         Annealing & 0.075368 &   0.042480 &    0.000488 & 0.000977 \\
            Census & 0.109863 &   0.016369 &    0.012207 & 0.009277 \\
     Contraceptive & 0.020996 &   0.000488 &    0.006836 & 0.000488 \\
Contraceptive (10) & 0.109863 &   0.000977 &    0.000488 & 0.001465 \\
Contraceptive (20) & 0.109863 &   0.004883 &    0.000488 & 0.002441 \\
Contraceptive (31) & 0.233398 &   0.004883 &    0.000488 & 0.002441 \\
Contraceptive (41) & 0.266113 &   0.012207 &    0.000488 & 0.034180 \\
         Covertype & 0.380371 &   0.518555 &    0.000488 & 0.518555 \\
   Credit Approval & 0.004883 &   0.061884 &    0.006836 & 0.003346 \\
     German Credit & 0.026855 &   0.034180 &    0.000977 & 0.063965 \\
German Credit (10) & 0.129395 &   0.622070 &    0.052246 & 0.000977 \\
German Credit (20) & 0.032854 &   0.077148 &    0.006836 & 0.002441 \\
German Credit (30) & 0.032854 &   0.012207 &    0.004883 & 0.000977 \\
German Credit (41) & 0.020795 &   0.006836 &    0.001465 & 0.006836 \\
     Heart Disease & 0.020996 &   0.000977 &    0.000488 & 0.000488 \\
Heart Disease (21) & 0.020996 &   0.000488 &    0.000488 & 0.000488 \\
\end{longtable}


\begin{longtable}{cccccc}
\caption{Adjusted p-values the Holm-Bonferroni test. Statistical significance is tested at a level of $lpha = 0.05$. The null hypothesis is that the benchmark methods do not perform better than the control method (G-SMOTENC).}
\label{tbl:holms_test}\\
\toprule
Classifier &                               Metric &      G-SMOTE &    RAND-OVER &  RAND-UNDER &      SMOTENC \\
\midrule
\endfirsthead
\caption[]{Adjusted p-values the Holm-Bonferroni test. Statistical significance is tested at a level of $lpha = 0.05$. The null hypothesis is that the benchmark methods do not perform better than the control method (G-SMOTENC).} \\
\toprule
Classifier &                               Metric &      G-SMOTE &    RAND-OVER &  RAND-UNDER &      SMOTENC \\
\midrule
\endhead
\midrule
\multicolumn{6}{r}{{Continued on next page}} \\
\midrule
\endfoot

\bottomrule
\endlastfoot
        DT &                   mean\_test\_accuracy & 1.648031e-01 & 9.608128e-07 &    0.000003 & 1.371459e-06 \\
        DT &                   mean\_test\_f1\_macro & 2.043336e-04 & 8.040421e-01 &    0.002443 & 1.452723e-01 \\
        DT & mean\_test\_geometric\_mean\_score\_macro & 4.476644e-07 & 1.742971e-05 &    0.000375 & 1.036473e-05 \\
       KNN &                   mean\_test\_accuracy & 4.622858e-05 & 5.759339e-06 &    0.000008 & 1.982851e-05 \\
       KNN &                   mean\_test\_f1\_macro & 8.534464e-04 & 7.155936e-02 &    0.021107 & 7.155936e-02 \\
       KNN & mean\_test\_geometric\_mean\_score\_macro & 1.306020e-05 & 5.700709e-04 &    0.000570 & 2.476249e-05 \\
        LR &                   mean\_test\_accuracy & 2.924409e-04 & 7.602412e-05 &    0.000019 & 2.924409e-04 \\
        LR &                   mean\_test\_f1\_macro & 3.591368e-03 & 1.036683e-02 &    0.009460 & 3.591368e-03 \\
        LR & mean\_test\_geometric\_mean\_score\_macro & 1.306117e-07 & 2.833456e-07 &    0.000005 & 2.833456e-07 \\
        RF &                   mean\_test\_accuracy & 1.733936e-02 & 5.073912e-05 &    0.000001 & 1.751144e-05 \\
        RF &                   mean\_test\_f1\_macro & 1.362126e-06 & 6.970889e-05 &    0.466845 & 6.970889e-05 \\
        RF & mean\_test\_geometric\_mean\_score\_macro & 3.053280e-06 & 1.067822e-05 &    0.000025 & 1.067822e-05 \\
\end{longtable}


\subsection{Discussion}~\label{sec:discussion}

\begin{figure}
	\centering
	\includegraphics[width=\linewidth]{../analysis/consistency_analysis_plot}
    \captionbelow{Average ranking of oversamplers over different
    characteristics of the datasets used in the experiment. Legend: IR ---
Imbalance Ratio, Classes --- Number of classes in the dataset, M/NM ratio ---
ratio between the number of metric and non-metric features, E$($F-Score$)$ ---
Mean F-Score of dataset across all combinations of classifiers and
oversamplers.}
\end{figure}

\section{Conclusion}~\label{sec:conclusion}

\bibliography{references}
\bibliographystyle{ieeetr}

\appendix

\section{Appendix}

\begin{longtable}{ccccccccc}
\caption{Classification performance after parameter tuning for eachdataset, classifier and oversampler.}
\label{tbl:wide_optimal}\\
\toprule
           Dataset & Classifier &  Metric &      G-SMOTENC &           NONE &        SMOTENC &            ROS &            RUS &      SMOTE-ENC \\
\midrule
\endfirsthead
\caption[]{Classification performance after parameter tuning for eachdataset, classifier and oversampler.} \\
\toprule
           Dataset & Classifier &  Metric &      G-SMOTENC &           NONE &        SMOTENC &            ROS &            RUS &      SMOTE-ENC \\
\midrule
\endhead
\midrule
\multicolumn{9}{r}{{Continued on next page}} \\
\midrule
\endfoot

\bottomrule
\endlastfoot
           Abalone &         DT &      OA &          0.221 & \textbf{0.256} &          0.190 &          0.203 &          0.207 &          0.191 \\
           Abalone &         DT & F-Score &          0.168 & \textbf{0.170} &          0.156 &          0.154 &          0.132 &          0.158 \\
           Abalone &         DT &  G-Mean & \textbf{0.460} &          0.413 &          0.445 &          0.457 &          0.421 &          0.443 \\
           Abalone &        KNN &      OA &          0.215 & \textbf{0.237} &          0.186 &          0.197 &          0.188 &          0.191 \\
           Abalone &        KNN & F-Score & \textbf{0.167} &          0.157 &          0.150 &          0.151 &          0.140 &          0.153 \\
           Abalone &        KNN &  G-Mean & \textbf{0.429} &          0.391 &          0.409 &          0.397 &          0.421 &          0.409 \\
           Abalone &         LR &      OA &          0.235 & \textbf{0.272} &          0.228 &          0.229 &          0.195 &          0.228 \\
           Abalone &         LR & F-Score & \textbf{0.189} &          0.180 &          0.186 &          0.179 &          0.166 &          0.182 \\
           Abalone &         LR &  G-Mean & \textbf{0.473} &          0.415 &          0.466 &          0.456 &          0.441 &          0.464 \\
           Abalone &         RF &      OA &          0.237 & \textbf{0.276} &          0.221 &          0.224 &          0.197 &          0.225 \\
           Abalone &         RF & F-Score & \textbf{0.194} &          0.174 &          0.180 &          0.184 &          0.162 &          0.180 \\
           Abalone &         RF &  G-Mean & \textbf{0.486} &          0.416 &          0.461 &          0.465 &          0.448 &          0.458 \\
             Adult &         DT &      OA &          0.830 & \textbf{0.835} &          0.785 &          0.800 &          0.785 &          0.781 \\
             Adult &         DT & F-Score & \textbf{0.767} &          0.763 &          0.754 &          0.755 &          0.744 &          0.749 \\
             Adult &         DT &  G-Mean & \textbf{0.809} &          0.747 &          0.808 &          0.806 &          0.801 &          0.799 \\
             Adult &        KNN &      OA &          0.786 & \textbf{0.805} &          0.781 &          0.763 &          0.761 &          0.767 \\
             Adult &        KNN & F-Score & \textbf{0.738} &          0.732 &          0.735 &          0.718 &          0.728 &          0.720 \\
             Adult &        KNN &  G-Mean &          0.766 &          0.724 &          0.762 &          0.757 & \textbf{0.780} &          0.752 \\
             Adult &         LR &      OA &          0.803 & \textbf{0.839} &          0.803 &          0.804 &          0.801 &          0.799 \\
             Adult &         LR & F-Score &          0.768 & \textbf{0.773} &          0.767 &          0.771 &          0.769 &          0.764 \\
             Adult &         LR &  G-Mean &          0.813 &          0.758 &          0.805 & \textbf{0.815} & \textbf{0.815} &          0.805 \\
             Adult &         RF &      OA &          0.820 & \textbf{0.832} &          0.757 &          0.755 &          0.753 &          0.761 \\
             Adult &         RF & F-Score & \textbf{0.769} &          0.739 &          0.727 &          0.729 &          0.728 &          0.732 \\
             Adult &         RF &  G-Mean &          0.796 &          0.711 &          0.787 & \textbf{0.797} & \textbf{0.797} &          0.793 \\
        Adult (10) &         DT &      OA & \textbf{0.930} &          0.928 &          0.822 &          0.789 &          0.775 &          0.819 \\
        Adult (10) &         DT & F-Score & \textbf{0.711} &          0.708 &          0.656 &          0.641 &          0.630 &          0.644 \\
        Adult (10) &         DT &  G-Mean &          0.812 &          0.663 &          0.807 & \textbf{0.815} &          0.808 &          0.788 \\
        Adult (10) &        KNN &      OA &          0.864 & \textbf{0.909} &          0.854 &          0.851 &          0.745 &          0.853 \\
        Adult (10) &        KNN & F-Score & \textbf{0.667} &          0.652 &          0.658 &          0.648 &          0.602 &          0.652 \\
        Adult (10) &        KNN &  G-Mean &          0.745 &          0.629 &          0.747 &          0.722 & \textbf{0.783} &          0.712 \\
        Adult (10) &         LR &      OA &          0.836 & \textbf{0.925} &          0.837 &          0.815 &          0.791 &          0.831 \\
        Adult (10) &         LR & F-Score &          0.666 & \textbf{0.705} &          0.667 &          0.663 &          0.647 &          0.665 \\
        Adult (10) &         LR &  G-Mean &          0.804 &          0.663 &          0.787 &          0.811 & \textbf{0.814} &          0.783 \\
        Adult (10) &         RF &      OA &          0.899 & \textbf{0.924} &          0.773 &          0.763 &          0.743 &          0.781 \\
        Adult (10) &         RF & F-Score & \textbf{0.718} &          0.615 &          0.620 &          0.624 &          0.610 &          0.626 \\
        Adult (10) &         RF &  G-Mean & \textbf{0.809} &          0.579 &          0.786 &          0.806 &          0.806 &          0.786 \\
         Annealing &         DT &      OA &          0.824 & \textbf{0.843} &          0.742 &          0.733 &          0.694 &          0.720 \\
         Annealing &         DT & F-Score & \textbf{0.736} &          0.643 &          0.732 &          0.724 &          0.683 &          0.718 \\
         Annealing &         DT &  G-Mean & \textbf{0.914} &          0.738 &          0.909 &          0.906 &          0.880 &          0.901 \\
         Annealing &        KNN &      OA &          0.849 &          0.847 &          0.829 & \textbf{0.854} &          0.508 &          0.830 \\
         Annealing &        KNN & F-Score &          0.780 &          0.724 &          0.747 & \textbf{0.783} &          0.476 &          0.741 \\
         Annealing &        KNN &  G-Mean &          0.901 &          0.781 &          0.867 & \textbf{0.909} &          0.814 &          0.856 \\
         Annealing &         LR &      OA &          0.572 & \textbf{0.814} &          0.573 &          0.566 &          0.510 &          0.552 \\
         Annealing &         LR & F-Score & \textbf{0.620} &          0.540 &          0.617 &          0.615 &          0.496 &          0.499 \\
         Annealing &         LR &  G-Mean & \textbf{0.851} &          0.663 &          0.843 &          0.848 &          0.811 &          0.821 \\
         Annealing &         RF &      OA & \textbf{0.868} & \textbf{0.868} &          0.729 &          0.733 &          0.637 &          0.759 \\
         Annealing &         RF & F-Score & \textbf{0.800} &          0.644 &          0.730 &          0.736 &          0.641 &          0.743 \\
         Annealing &         RF &  G-Mean & \textbf{0.917} &          0.727 &          0.904 &          0.910 &          0.873 &          0.887 \\
            Census &         DT &      OA &          0.942 & \textbf{0.943} &          0.894 &          0.844 &          0.795 &          0.293 \\
            Census &         DT & F-Score & \textbf{0.733} &          0.731 &          0.693 &          0.652 &          0.617 &          0.258 \\
            Census &         DT &  G-Mean &          0.813 &          0.698 &          0.800 &          0.814 & \textbf{0.817} &          0.621 \\
            Census &        KNN &      OA &          0.874 & \textbf{0.933} &          0.867 &          0.878 &          0.731 &          0.871 \\
            Census &        KNN & F-Score &          0.652 &          0.648 & \textbf{0.655} &          0.640 &          0.567 &          0.641 \\
            Census &        KNN &  G-Mean &          0.767 &          0.620 &          0.768 &          0.733 & \textbf{0.794} &          0.740 \\
            Census &         LR &      OA &          0.940 & \textbf{0.949} &          0.938 &          0.940 &          0.815 &          0.828 \\
            Census &         LR & F-Score &          0.760 &          0.743 &          0.760 & \textbf{0.762} &          0.639 &          0.630 \\
            Census &         LR &  G-Mean &          0.807 &          0.707 &          0.782 &          0.801 & \textbf{0.837} &          0.794 \\
            Census &         RF &      OA &          0.876 & \textbf{0.933} &          0.819 &          0.740 &          0.714 &          0.799 \\
            Census &         RF & F-Score & \textbf{0.679} &          0.483 &          0.636 &          0.580 &          0.562 &          0.614 \\
            Census &         RF &  G-Mean & \textbf{0.827} &          0.500 &          0.818 &          0.822 &          0.814 &          0.810 \\
     Contraceptive &         DT &      OA & \textbf{0.563} &          0.538 &          0.537 &          0.512 &          0.525 &          0.528 \\
     Contraceptive &         DT & F-Score & \textbf{0.549} &          0.518 &          0.529 &          0.507 &          0.520 &          0.521 \\
     Contraceptive &         DT &  G-Mean & \textbf{0.661} &          0.630 &          0.646 &          0.630 &          0.641 &          0.638 \\
     Contraceptive &        KNN &      OA &          0.465 & \textbf{0.478} &          0.455 &          0.435 &          0.468 &          0.461 \\
     Contraceptive &        KNN & F-Score &          0.460 & \textbf{0.462} &          0.450 &          0.432 &          0.461 &          0.455 \\
     Contraceptive &        KNN &  G-Mean &          0.588 &          0.580 &          0.579 &          0.566 & \textbf{0.590} &          0.583 \\
     Contraceptive &         LR &      OA & \textbf{0.515} &          0.514 &          0.514 &          0.510 &          0.510 &          0.513 \\
     Contraceptive &         LR & F-Score & \textbf{0.512} &          0.492 &          0.509 &          0.505 &          0.506 &          0.508 \\
     Contraceptive &         LR &  G-Mean & \textbf{0.635} &          0.604 &          0.631 &          0.628 &          0.627 &          0.630 \\
     Contraceptive &         RF &      OA &          0.553 & \textbf{0.557} &          0.540 &          0.534 &          0.526 &          0.536 \\
     Contraceptive &         RF & F-Score & \textbf{0.545} &          0.524 &          0.535 &          0.529 &          0.522 &          0.530 \\
     Contraceptive &         RF &  G-Mean & \textbf{0.659} &          0.634 &          0.653 &          0.649 &          0.643 &          0.649 \\
Contraceptive (10) &         DT &      OA & \textbf{0.645} & \textbf{0.645} &          0.568 &          0.528 &          0.487 &          0.592 \\
Contraceptive (10) &         DT & F-Score &          0.479 &          0.452 &          0.478 &          0.454 &          0.414 & \textbf{0.490} \\
Contraceptive (10) &         DT &  G-Mean &          0.644 &          0.584 & \textbf{0.648} &          0.637 &          0.610 & \textbf{0.648} \\
Contraceptive (10) &        KNN &      OA &          0.524 & \textbf{0.570} &          0.508 &          0.495 &          0.451 &          0.512 \\
Contraceptive (10) &        KNN & F-Score & \textbf{0.419} &          0.404 &          0.410 &          0.404 &          0.368 &          0.413 \\
Contraceptive (10) &        KNN &  G-Mean & \textbf{0.576} &          0.529 &          0.561 &          0.569 &          0.561 &          0.563 \\
Contraceptive (10) &         LR &      OA &          0.516 & \textbf{0.622} &          0.506 &          0.489 &          0.476 &          0.503 \\
Contraceptive (10) &         LR & F-Score & \textbf{0.431} &          0.375 &          0.426 &          0.425 &          0.411 & \textbf{0.431} \\
Contraceptive (10) &         LR &  G-Mean &          0.619 &          0.526 &          0.609 & \textbf{0.624} &          0.618 &          0.621 \\
Contraceptive (10) &         RF &      OA &          0.648 & \textbf{0.651} &          0.569 &          0.550 &          0.494 &          0.573 \\
Contraceptive (10) &         RF & F-Score & \textbf{0.500} &          0.387 &          0.473 &          0.471 &          0.425 &          0.480 \\
Contraceptive (10) &         RF &  G-Mean & \textbf{0.656} &          0.542 &          0.639 &          0.650 &          0.625 &          0.646 \\
Contraceptive (20) &         DT &      OA & \textbf{0.671} &          0.659 &          0.612 &          0.556 &          0.456 &          0.620 \\
Contraceptive (20) &         DT & F-Score & \textbf{0.475} &          0.430 &          0.459 &          0.428 &          0.371 &          0.470 \\
Contraceptive (20) &         DT &  G-Mean &          0.643 &          0.570 &          0.626 &          0.632 &          0.605 & \textbf{0.645} \\
Contraceptive (20) &        KNN &      OA &          0.556 & \textbf{0.600} &          0.529 &          0.541 &          0.442 &          0.543 \\
Contraceptive (20) &        KNN & F-Score & \textbf{0.399} &          0.375 &          0.384 &          0.389 &          0.345 &          0.395 \\
Contraceptive (20) &        KNN &  G-Mean & \textbf{0.565} &          0.519 &          0.544 &          0.537 &          0.549 &          0.556 \\
Contraceptive (20) &         LR &      OA &          0.506 & \textbf{0.641} &          0.508 &          0.486 &          0.440 &          0.514 \\
Contraceptive (20) &         LR & F-Score & \textbf{0.397} &          0.375 & \textbf{0.397} &          0.389 &          0.358 &          0.393 \\
Contraceptive (20) &         LR &  G-Mean &          0.608 &          0.523 &          0.604 & \textbf{0.613} &          0.585 &          0.597 \\
Contraceptive (20) &         RF &      OA &          0.668 & \textbf{0.674} &          0.588 &          0.562 &          0.475 &          0.605 \\
Contraceptive (20) &         RF & F-Score & \textbf{0.473} &          0.384 &          0.450 &          0.436 &          0.389 &          0.454 \\
Contraceptive (20) &         RF &  G-Mean &          0.659 &          0.535 &          0.641 & \textbf{0.670} &          0.633 &          0.642 \\
Contraceptive (31) &         DT &      OA &          0.667 & \textbf{0.670} &          0.608 &          0.604 &          0.440 &          0.644 \\
Contraceptive (31) &         DT & F-Score & \textbf{0.454} &          0.441 &          0.438 &          0.453 &          0.346 & \textbf{0.454} \\
Contraceptive (31) &         DT &  G-Mean &          0.642 &          0.577 &          0.605 & \textbf{0.655} &          0.592 &          0.629 \\
Contraceptive (31) &        KNN &      OA &          0.563 & \textbf{0.633} &          0.545 &          0.550 &          0.405 &          0.548 \\
Contraceptive (31) &        KNN & F-Score & \textbf{0.403} &          0.385 &          0.384 &          0.378 &          0.298 &          0.387 \\
Contraceptive (31) &        KNN &  G-Mean & \textbf{0.574} &          0.527 &          0.544 &          0.531 &          0.511 &          0.555 \\
Contraceptive (31) &         LR &      OA &          0.500 & \textbf{0.656} &          0.508 &          0.483 &          0.423 &          0.516 \\
Contraceptive (31) &         LR & F-Score & \textbf{0.379} &          0.376 & \textbf{0.379} &          0.374 &          0.336 & \textbf{0.379} \\
Contraceptive (31) &         LR &  G-Mean & \textbf{0.597} &          0.523 &          0.579 &          0.585 &          0.580 &          0.574 \\
Contraceptive (31) &         RF &      OA &          0.681 & \textbf{0.683} &          0.608 &          0.583 &          0.442 &          0.616 \\
Contraceptive (31) &         RF & F-Score &          0.450 &          0.378 &          0.434 &          0.435 &          0.349 & \textbf{0.452} \\
Contraceptive (31) &         RF &  G-Mean & \textbf{0.647} &          0.531 &          0.630 &          0.640 &          0.600 &          0.626 \\
Contraceptive (41) &         DT &      OA &          0.651 & \textbf{0.666} &          0.588 &          0.566 &          0.433 &          0.589 \\
Contraceptive (41) &         DT & F-Score & \textbf{0.459} &          0.426 &          0.408 &          0.409 &          0.336 &          0.416 \\
Contraceptive (41) &         DT &  G-Mean & \textbf{0.622} &          0.573 &          0.579 &          0.589 &          0.555 &          0.589 \\
Contraceptive (41) &        KNN &      OA &          0.563 & \textbf{0.611} &          0.546 &          0.538 &          0.395 &          0.541 \\
Contraceptive (41) &        KNN & F-Score & \textbf{0.393} &          0.373 &          0.381 &          0.370 &          0.289 &          0.373 \\
Contraceptive (41) &        KNN &  G-Mean &          0.542 &          0.515 & \textbf{0.550} &          0.526 &          0.515 &          0.531 \\
Contraceptive (41) &         LR &      OA &          0.525 & \textbf{0.658} &          0.524 &          0.504 &          0.435 &          0.530 \\
Contraceptive (41) &         LR & F-Score &          0.389 &          0.375 & \textbf{0.393} &          0.387 &          0.336 & \textbf{0.393} \\
Contraceptive (41) &         LR &  G-Mean &          0.606 &          0.520 &          0.604 & \textbf{0.627} &          0.569 &          0.600 \\
Contraceptive (41) &         RF &      OA &          0.665 & \textbf{0.681} &          0.598 &          0.588 &          0.415 &          0.596 \\
Contraceptive (41) &         RF & F-Score & \textbf{0.444} &          0.378 &          0.418 &          0.429 &          0.323 &          0.416 \\
Contraceptive (41) &         RF &  G-Mean &          0.612 &          0.528 & \textbf{0.616} & \textbf{0.616} &          0.566 &          0.608 \\
         Covertype &         DT &      OA &          0.580 & \textbf{0.705} &          0.587 &          0.567 &          0.450 &          0.552 \\
         Covertype &         DT & F-Score &          0.484 & \textbf{0.490} &          0.481 &          0.475 &          0.361 &          0.474 \\
         Covertype &         DT &  G-Mean & \textbf{0.769} &          0.671 &          0.758 &          0.758 &          0.700 &          0.751 \\
         Covertype &        KNN &      OA &          0.690 & \textbf{0.700} &          0.683 &          0.699 &          0.454 &          0.636 \\
         Covertype &        KNN & F-Score &          0.532 &          0.457 &          0.535 & \textbf{0.561} &          0.367 &          0.484 \\
         Covertype &        KNN &  G-Mean &          0.745 &          0.642 &          0.753 & \textbf{0.763} &          0.691 &          0.744 \\
         Covertype &         LR &      OA &          0.637 & \textbf{0.721} &          0.640 &          0.611 &          0.472 &          0.617 \\
         Covertype &         LR & F-Score &          0.516 &          0.507 & \textbf{0.526} &          0.492 &          0.353 &          0.429 \\
         Covertype &         LR &  G-Mean & \textbf{0.792} &          0.678 &          0.786 &          0.790 &          0.697 &          0.725 \\
         Covertype &         RF &      OA &          0.598 & \textbf{0.704} &          0.583 &          0.587 &          0.485 &          0.338 \\
         Covertype &         RF & F-Score &          0.517 &          0.360 &          0.507 & \textbf{0.519} &          0.394 &          0.284 \\
         Covertype &         RF &  G-Mean &          0.800 &          0.572 &          0.799 & \textbf{0.804} &          0.737 &          0.691 \\
   Credit Approval &         DT &      OA & \textbf{0.867} &          0.847 &          0.862 &          0.861 &          0.865 &          0.862 \\
   Credit Approval &         DT & F-Score & \textbf{0.867} &          0.845 &          0.862 &          0.861 &          0.865 &          0.862 \\
   Credit Approval &         DT &  G-Mean & \textbf{0.874} &          0.848 &          0.869 &          0.867 &          0.872 &          0.869 \\
   Credit Approval &        KNN &      OA & \textbf{0.870} &          0.865 &          0.868 & \textbf{0.870} &          0.865 &          0.867 \\
   Credit Approval &        KNN & F-Score & \textbf{0.869} &          0.864 &          0.867 & \textbf{0.869} &          0.864 &          0.866 \\
   Credit Approval &        KNN &  G-Mean & \textbf{0.871} &          0.865 &          0.868 & \textbf{0.871} &          0.866 &          0.867 \\
   Credit Approval &         LR &      OA &          0.873 &          0.868 &          0.871 & \textbf{0.874} &          0.873 &          0.873 \\
   Credit Approval &         LR & F-Score &          0.873 &          0.868 &          0.871 & \textbf{0.874} &          0.873 &          0.873 \\
   Credit Approval &         LR &  G-Mean &          0.877 &          0.873 &          0.877 & \textbf{0.879} &          0.878 &          0.878 \\
   Credit Approval &         RF &      OA &          0.876 & \textbf{0.877} &          0.871 &          0.868 &          0.868 &          0.873 \\
   Credit Approval &         RF & F-Score &          0.876 & \textbf{0.877} &          0.871 &          0.868 &          0.868 &          0.872 \\
   Credit Approval &         RF &  G-Mean & \textbf{0.879} & \textbf{0.879} &          0.876 &          0.872 &          0.873 &          0.875 \\
     German Credit &         DT &      OA &          0.704 & \textbf{0.713} &          0.702 &          0.660 &          0.644 &          0.701 \\
     German Credit &         DT & F-Score &          0.662 &          0.608 &          0.654 &          0.633 &          0.623 & \textbf{0.664} \\
     German Credit &         DT &  G-Mean & \textbf{0.681} &          0.608 &          0.667 &          0.663 &          0.660 &          0.678 \\
     German Credit &        KNN &      OA &          0.681 & \textbf{0.718} &          0.682 &          0.670 &          0.641 &          0.657 \\
     German Credit &        KNN & F-Score & \textbf{0.653} &          0.628 &          0.650 &          0.636 &          0.616 &          0.626 \\
     German Credit &        KNN &  G-Mean & \textbf{0.675} &          0.621 &          0.668 &          0.656 &          0.642 &          0.646 \\
     German Credit &         LR &      OA &          0.727 & \textbf{0.751} &          0.729 &          0.724 &          0.712 &          0.713 \\
     German Credit &         LR & F-Score &          0.695 &          0.681 & \textbf{0.697} & \textbf{0.697} &          0.686 &          0.676 \\
     German Credit &         LR &  G-Mean & \textbf{0.722} &          0.672 &          0.713 &          0.720 &          0.713 &          0.696 \\
     German Credit &         RF &      OA & \textbf{0.760} &          0.741 &          0.739 &          0.737 &          0.700 &          0.726 \\
     German Credit &         RF & F-Score &          0.701 &          0.580 &          0.702 & \textbf{0.709} &          0.680 &          0.688 \\
     German Credit &         RF &  G-Mean &          0.715 &          0.588 &          0.716 & \textbf{0.730} &          0.719 &          0.699 \\
German Credit (10) &         DT &      OA & \textbf{0.909} &          0.906 &          0.804 &          0.713 &          0.696 &          0.752 \\
German Credit (10) &         DT & F-Score & \textbf{0.575} &          0.539 &          0.572 &          0.526 &          0.511 &          0.539 \\
German Credit (10) &         DT &  G-Mean &          0.628 &          0.535 &          0.629 & \textbf{0.644} &          0.631 &          0.593 \\
German Credit (10) &        KNN &      OA &          0.787 & \textbf{0.913} &          0.757 &          0.835 &          0.684 &          0.795 \\
German Credit (10) &        KNN & F-Score &          0.578 & \textbf{0.581} &          0.558 &          0.573 &          0.528 &          0.560 \\
German Credit (10) &        KNN &  G-Mean &          0.662 &          0.559 &          0.643 &          0.588 & \textbf{0.667} &          0.597 \\
German Credit (10) &         LR &      OA &          0.839 & \textbf{0.904} &          0.831 &          0.799 &          0.682 &          0.829 \\
German Credit (10) &         LR & F-Score &          0.619 &          0.596 &          0.610 & \textbf{0.620} &          0.550 & \textbf{0.620} \\
German Credit (10) &         LR &  G-Mean &          0.683 &          0.578 &          0.675 &          0.716 & \textbf{0.722} &          0.681 \\
German Credit (10) &         RF &      OA & \textbf{0.910} &          0.909 &          0.865 &          0.877 &          0.696 &          0.860 \\
German Credit (10) &         RF & F-Score &          0.624 &          0.476 &          0.614 & \textbf{0.661} &          0.557 &          0.610 \\
German Credit (10) &         RF &  G-Mean &          0.653 &          0.500 &          0.646 &          0.709 & \textbf{0.729} &          0.628 \\
German Credit (20) &         DT &      OA & \textbf{0.952} & \textbf{0.952} &          0.875 &          0.795 &          0.668 &          0.880 \\
German Credit (20) &         DT & F-Score &          0.573 &          0.525 &          0.559 &          0.522 &          0.457 & \textbf{0.579} \\
German Credit (20) &         DT &  G-Mean &          0.666 &          0.529 &          0.679 & \textbf{0.690} &          0.629 &          0.674 \\
German Credit (20) &        KNN &      OA &          0.856 & \textbf{0.952} &          0.826 &          0.905 &          0.679 &          0.872 \\
German Credit (20) &        KNN & F-Score & \textbf{0.561} &          0.535 &          0.528 &          0.556 &          0.491 &          0.538 \\
German Credit (20) &        KNN &  G-Mean &          0.692 &          0.527 &          0.635 &          0.570 & \textbf{0.709} &          0.601 \\
German Credit (20) &         LR &      OA &          0.913 & \textbf{0.952} &          0.910 &          0.838 &          0.680 &          0.891 \\
German Credit (20) &         LR & F-Score & \textbf{0.596} &          0.534 &          0.593 &          0.553 &          0.473 &          0.568 \\
German Credit (20) &         LR &  G-Mean &          0.651 &          0.531 &          0.627 &          0.661 & \textbf{0.682} &          0.616 \\
German Credit (20) &         RF &      OA & \textbf{0.954} &          0.952 &          0.920 &          0.931 &          0.709 &          0.920 \\
German Credit (20) &         RF & F-Score & \textbf{0.597} &          0.488 &          0.574 &          0.572 &          0.493 &          0.576 \\
German Credit (20) &         RF &  G-Mean &          0.681 &          0.500 &          0.625 &          0.674 & \textbf{0.691} &          0.639 \\
German Credit (30) &         DT &      OA & \textbf{0.968} &          0.963 &          0.885 &          0.856 &          0.628 &          0.888 \\
German Credit (30) &         DT & F-Score & \textbf{0.558} &          0.509 &          0.526 &          0.506 &          0.413 &          0.528 \\
German Credit (30) &         DT &  G-Mean & \textbf{0.686} &          0.509 &          0.631 &          0.602 &          0.565 &          0.609 \\
German Credit (30) &        KNN &      OA &          0.902 & \textbf{0.968} &          0.849 &          0.935 &          0.697 &          0.900 \\
German Credit (30) &        KNN & F-Score & \textbf{0.530} &          0.492 &          0.512 &          0.519 &          0.473 &          0.507 \\
German Credit (30) &        KNN &  G-Mean &          0.681 &          0.500 &          0.588 &          0.536 & \textbf{0.705} &          0.536 \\
German Credit (30) &         LR &      OA &          0.921 & \textbf{0.967} &          0.918 &          0.877 &          0.611 &          0.920 \\
German Credit (30) &         LR & F-Score & \textbf{0.578} &          0.516 &          0.577 &          0.537 &          0.421 &          0.571 \\
German Credit (30) &         LR &  G-Mean &          0.649 &          0.510 &          0.650 & \textbf{0.661} &          0.660 &          0.608 \\
German Credit (30) &         RF &      OA & \textbf{0.968} & \textbf{0.968} &          0.942 &          0.954 &          0.705 &          0.947 \\
German Credit (30) &         RF & F-Score & \textbf{0.592} &          0.492 &          0.563 &          0.589 &          0.474 &          0.560 \\
German Credit (30) &         RF &  G-Mean & \textbf{0.689} &          0.500 &          0.601 &          0.606 &          0.679 &          0.618 \\
German Credit (41) &         DT &      OA & \textbf{0.976} &          0.971 &          0.916 &          0.905 &          0.635 &          0.898 \\
German Credit (41) &         DT & F-Score & \textbf{0.563} &          0.493 &          0.544 &          0.502 &          0.408 &          0.552 \\
German Credit (41) &         DT &  G-Mean & \textbf{0.636} &          0.497 &          0.615 &          0.520 &          0.524 &          0.626 \\
German Credit (41) &        KNN &      OA &          0.929 & \textbf{0.976} &          0.876 &          0.944 &          0.674 &          0.920 \\
German Credit (41) &        KNN & F-Score & \textbf{0.524} &          0.494 &          0.500 &          0.502 &          0.440 &          0.493 \\
German Credit (41) &        KNN &  G-Mean &          0.593 &          0.500 &          0.558 &          0.516 & \textbf{0.630} &          0.504 \\
German Credit (41) &         LR &      OA &          0.940 & \textbf{0.976} &          0.943 &          0.927 &          0.641 &          0.932 \\
German Credit (41) &         LR & F-Score &          0.546 &          0.494 & \textbf{0.552} &          0.515 &          0.420 &          0.516 \\
German Credit (41) &         LR &  G-Mean & \textbf{0.602} &          0.500 &          0.592 &          0.598 &          0.597 &          0.521 \\
German Credit (41) &         RF &      OA & \textbf{0.976} & \textbf{0.976} &          0.961 &          0.969 &          0.636 &          0.962 \\
German Credit (41) &         RF & F-Score & \textbf{0.598} &          0.494 &          0.566 &          0.591 &          0.413 &          0.561 \\
German Credit (41) &         RF &  G-Mean &          0.621 &          0.500 & \textbf{0.622} &          0.614 &          0.572 &          0.616 \\
     Heart Disease &         DT &      OA &          0.532 & \textbf{0.566} &          0.509 &          0.473 &          0.430 &          0.509 \\
     Heart Disease &         DT & F-Score & \textbf{0.371} &          0.322 &          0.342 &          0.331 &          0.295 &          0.339 \\
     Heart Disease &         DT &  G-Mean & \textbf{0.588} &          0.534 &          0.563 &          0.545 &          0.515 &          0.548 \\
     Heart Disease &        KNN &      OA &          0.538 & \textbf{0.564} &          0.535 &          0.534 &          0.504 &          0.528 \\
     Heart Disease &        KNN & F-Score & \textbf{0.363} &          0.287 &          0.360 &          0.352 &          0.341 &          0.348 \\
     Heart Disease &        KNN &  G-Mean & \textbf{0.571} &          0.509 & \textbf{0.571} &          0.560 &          0.557 &          0.557 \\
     Heart Disease &         LR &      OA &          0.558 & \textbf{0.584} &          0.557 &          0.536 &          0.480 &          0.562 \\
     Heart Disease &         LR & F-Score &          0.397 &          0.329 &          0.395 &          0.374 &          0.333 & \textbf{0.400} \\
     Heart Disease &         LR &  G-Mean &          0.601 &          0.539 &          0.601 &          0.603 &          0.567 & \textbf{0.610} \\
     Heart Disease &         RF &      OA &          0.553 & \textbf{0.601} &          0.546 &          0.539 &          0.480 &          0.555 \\
     Heart Disease &         RF & F-Score & \textbf{0.385} &          0.314 &          0.366 &          0.360 &          0.326 &          0.378 \\
     Heart Disease &         RF &  G-Mean & \textbf{0.600} &          0.531 &          0.580 &          0.569 &          0.566 &          0.582 \\
Heart Disease (21) &         DT &      OA &          0.532 & \textbf{0.566} &          0.512 &          0.486 &          0.431 &          0.510 \\
Heart Disease (21) &         DT & F-Score & \textbf{0.376} &          0.296 &          0.341 &          0.336 &          0.311 &          0.342 \\
Heart Disease (21) &         DT &  G-Mean & \textbf{0.598} &          0.509 &          0.558 &          0.562 &          0.538 &          0.551 \\
Heart Disease (21) &        KNN &      OA &          0.561 & \textbf{0.569} &          0.543 &          0.541 &          0.491 &          0.550 \\
Heart Disease (21) &        KNN & F-Score & \textbf{0.385} &          0.312 &          0.365 &          0.363 &          0.334 &          0.365 \\
Heart Disease (21) &        KNN &  G-Mean & \textbf{0.589} &          0.520 &          0.570 &          0.566 &          0.546 &          0.570 \\
Heart Disease (21) &         LR &      OA &          0.573 & \textbf{0.592} &          0.565 &          0.547 &          0.525 &          0.561 \\
Heart Disease (21) &         LR & F-Score & \textbf{0.408} &          0.331 &          0.405 &          0.387 &          0.343 &          0.405 \\
Heart Disease (21) &         LR &  G-Mean & \textbf{0.638} &          0.540 &          0.610 &          0.602 &          0.583 &          0.627 \\
Heart Disease (21) &         RF &      OA &          0.577 & \textbf{0.608} &          0.565 &          0.561 &          0.517 &          0.561 \\
Heart Disease (21) &         RF & F-Score & \textbf{0.417} &          0.323 &          0.390 &          0.383 &          0.337 &          0.386 \\
Heart Disease (21) &         RF &  G-Mean & \textbf{0.621} &          0.536 &          0.596 &          0.593 &          0.567 &          0.590 \\
\end{longtable}


\end{document}
