\documentclass[parskip=full]{scrartcl}

\pdfoutput=1

\title{Geometric SMOTENC \\ \LARGE{A geometrically enhanced drop-in
replacement for SMOTENC}}

\author{%
	Joao Fonseca\(^{1*}\), Georgios Douzas\(^{1}\), Fernando Bacao\(^{1}\)
	\\
	\small{\(^{1}\)NOVA Information Management School, Universidade Nova de Lisboa}
	\\
	\small{*Corresponding Author}
	\\
	\\
	\small{Postal Address: NOVA Information Management School, Campus de
    Campolide, 1070--312 Lisboa, Portugal}
	\\
	\small{Telephone: +351 21 382 8610}
}

\usepackage{breakcites}
\usepackage{float}
\usepackage{graphicx}
\usepackage{geometry}
\geometry{%
	a4paper,
	left=18mm,
	right=18mm,
	top=8mm,
}
\usepackage{amsmath}
\usepackage{enumitem}
\usepackage[ruled,vlined]{algorithm2e}
\usepackage{booktabs}
\usepackage{pgfplotstable}
\pgfplotsset{compat=1.14}
\usepackage{longtable}
\usepackage{tabu}
\usepackage{hyperref}
\date{}

\begin{document}

\maketitle

\begin{abstract}
    This is an abstract.
\end{abstract}

\section{Introduction}

This is text~\cite{Chawla2002}.

%\begin{figure}[H]
%	\centering
%	\includegraphics[width=12cm, keepaspectratio]{../analysis/fig3}
%    \captionbelow{An instance belonging to a minority class cluster and one of
%    its 5-nearest neighbors are selected. An observation belonging to the same
%    cluster is generated.}
%\end{figure}

 \begin{table}
     \centering
     \pgfplotstabletypeset[
         col sep=comma,
         string type,
         every head row/.style={%
             before row=\toprule,
             after row=\midrule
         },
         every last row/.style={after row=\bottomrule},
     ]{../analysis/datasets_description.csv}
     \caption{\label{tab:datasets_description}
         Description of the datasets collected after data preprocessing. The
         sampling strategy is similar across datasets. Legend: (IR) Imbalance
         Ratio
     }
\end{table}

\bibliography{references}
\bibliographystyle{ieeetr}

\end{document}
