\section{Conclusions}\label{sec:conclusions}

This literature review analyses various synthetic data generation-based
algorithms for tabular data, with a focus on external\hl{-}level applications.
Since synthetic data generation is a crucial step for various ML applications
and domains, it is essential to understand and compare which techniques and
types of algorithms are used for each of these problems. The usage of
synthetic data is an effective approach to better prepare datasets and ML
pipelines for a wide range of applications and/or address privacy concerns.
Our work proposed a taxonomy based on four key characteristics of generation
algorithms, which was used to characterize 70 data generation algorithms
across six ML problems. This analysis resulted in the categorization and
description of the generation mechanisms underlying each of the selected
algorithms into six main categories. Finally, we discussed several techniques
to evaluate synthetic data, as well as general recommendations and research
gaps based on the insights collected throughout the analysis of the
literature.

Despite the extensive research developed on \hl{several} methods for synthetic
data generation, there are still open questions regarding the theoretical
underpinnings of synthetic data adoption for each of the techniques, as well
as limitations in the different types of generation mechanisms and evaluation
procedures. However, the empirical work presented in the literature show\hl{s}
significant performance improvements and promising research directions for
future work.
