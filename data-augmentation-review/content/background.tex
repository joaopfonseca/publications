\section{Background}\label{sec:background}

In this section\hl{,} we define basic\hl{ }concepts, common goals,
trade-offs\hl{,} and
motivations regarding the generation of synthetic data in ML\@. We define
synthetic data generation as the production of artificial observations that
resemble naturally occurring ones within a certain domain, using a generative
model. It requires access to a training dataset, a generative process, or a
data stream. However, the constraints imposed \hl{on} this process largely
depend\hl{ } on the target ML task. For example, to generate artificial data
for regularization purposes in supervised learning (\textit{i.e.}, data
augmentation) the training dataset must be annotated. The production of
anonymized datasets using synthetic data generation requires synthetic
datasets to be different from the original data while following similar
statistical properties. Domain knowledge may also be necessary to encode
specific relationships among features into the generative process.


\subsection{Relevant Learning Problems}

%% Anonymize datasets (differential privacy)
The breach of sensitive information is an important barrier to the sharing of
datasets, especially when it concerns personal
information~\cite{dankar2021fake}. One solution for this problem is the
generation of synthetic data without identifiable information. Generally
speaking, ML tasks that require data with sensitive information are not
compromised when using synthetic data. The experiment conducted by
\cite{patki2016synthetic} using relational datasets showed that in 11 out
\hl{of} 15 comparisons ($\approx 73\%$), practitioners performing predictive
modelling tasks using fully synthetic datasets performed the same or better
than those using the original dataset. Optionally, anonymized synthetic data
may be produced with theoretical privacy guarantees, using differential
privacy techniques. This topic is discussed in Section~\ref{sec:data-privacy}.

%% Generalization/regularization
A common problem in the training of \hl{ML classifiers is} their
capacity to generalize~\cite{Zhang2021} (\textit{i.e.}, reduce the difference
in classification performance between known and unseen observations). \hl{This
is particularly true for deep neural networks since they require the
estimation of high amounts of parameters.} Data augmentation is a common
method to address this problem for any type of ML classifier. The generation
of synthetic observations increases the range of the input space used in the
training phase and reduces the difference in performance between known and new
observations. Although other regularization methods exist, data augmentation
is a useful method since it does not affect the choice in the architecture of
the ML classifier and does not exclude the usage of other regularization
methods. In domains such as computer vision and NLP, data augmentation is also
used to improve the robustness of models against adversarial
attacks~\cite{zeng2020data, morris2020textattack}. These topics are discussed
\hl{in} higher detail in Section~\ref{sec:regularization}.

%% Oversampling
In supervised learning, synthetic data generation is often motivated by the
need to balance target class distributions (\textit{i.e.}, oversampling).
Since most ML classifiers are designed to perform best with balanced datasets,
defining an appropriate decision boundary to distinguish rare classes becomes
difficult~\cite{saez2016analyzing}. Although there are other approaches to
address imbalanced learning, oversampling techniques are generally easier to
implement since they do not involve modifications to the classifier. This
topic is discussed \hl{in} higher detail in Section~\ref{sec:oversampling}.

%% Active Learning + Few-shot Learning
In supervised learning tasks where labeled data is not readily available, but
can be labeled, an Active Learning (AL) method may be used to improve the
efficiency of the labeling process. AL aims to reduce the cost of producing
training datasets by finding the most informative observations to label and
feed into the classifier~\cite{fonseca2021increasing}. In this case, the
generation of synthetic data is particularly useful to reduce the amount of
labeled data required for a successful ML project. This topic is discussed in
Section~\ref{sec:active-learning}.

%% Semi-supervised Learning + Self-supervised learning
Two other techniques reliant on synthetic data generation are Semi-supervised
Learning (Semi-SL) and Self-Supervised Learning (Self-SL). The former
leverages both labeled and unlabeled data in the training phase,
simultaneously, while several methods apply perturbations on the training data
as part of the training procedure~\cite{van2020survey}. The latter, Self-SL,
is a technique used to train neural networks in the absence of labeled data.
Several Semi-SL and Self-SL methods use synthetic data generation as a core
element. These methods are discussed in
Sections~\ref{sec:semi-supervised-learning}
and~\ref{sec:self-supervised-learning}.


% Concepts definition, goals + trade-offs, Motivations

% Concepts definition
% - Original dataset
% - Synthetic dataset
% - Generator
% - Quality criteria (e.g., similarity function)
% - Objective performance metric (ML method to improve)
 
% Common success criteria 
% - The synthetic dataset should be different than the original dataset (i.e.,
%   no duplicates)
% - It should minimize a similarity function
% - The statistical properties of the dataset are preserved
% - It should improve the output of the objective performance
 
% Trade-offs
% - Additional computational overhead
% - Expose the classifier to unfeasible input areas

\subsection{Problem Formulation}\label{sec:problem-formulation}

The original dataset, $\mathcal{D} = \mathcal{D}_L \cup \mathcal{D}_U$,
is a collection of real observations and is distinguished according to whether
a target feature exists, $\mathcal{D}_L = {((x_i, y_i))}^l_{i=1}$, or not,
$\mathcal{D}_U = {(x_i)}^{u}_{i=1}$. All three datasets, $\mathcal{D}$,
$\mathcal{D}_L$ and $\mathcal{D}_U$ consist of ordered collections with
lengths $l+u$, $l$ and $u$, respectively. Synthetic data generation is
performed using a generator, $f_{gen}(x;\tau) = x^s$, where $\tau$
defines the generation policy (\textit{i.e.}, its hyperparameters), $x \in
\mathcal{D}$ is an observation and $x^s \in \mathcal{D}^s$ is a
synthetic observation. Analogous to $\mathcal{D}$, the synthetic dataset,
$\mathcal{D}^s$, is also distinguished according to whether there is an
assignment of a target feature, $\mathcal{D}^s_L = {((x^s_j,
y^s_j))}^{l'}_{j=1}$, or not, $\mathcal{D}^s_U =
{(x^s_j)}^{u'}_{j=1}$.

% - Quality criteria (e.g., similarity function)
Depending on the ML task, it may be relevant to establish metrics to measure
the quality of $\mathcal{D}^s$. In this case, a metric
$f_{qual}(\mathcal{D}^s, \mathcal{D})$ is used to determine the level of
similarity/dissimilarity between $\mathcal{D}$ and $\mathcal{D}^s$. In
addition, a performance metric to estimate the performance of a model on the
objective task, $f_{per}$, may be used to determine the appropriateness of a
model with parameters $\theta$, \textit{i.e.}, $f_{\theta}$. The generator's
goal is to generate $\mathcal{D}^s$ with arbitrary length, given
$\mathcal{D} \sim \mathbb{P}$ and $\mathcal{D}^s \sim \mathbb{P}^s$, such
that $\mathbb{P}^s \approx \mathbb{P}$, $x_i \neq x_j \forall x_i \in
\mathcal{D} \wedge x_j \in \mathcal{D}^s$. $f_{gen}(x;\tau)$ attempts to
generate a $\mathcal{D}^s$ that maximizes either $f_{per}$, $f_{qual}$, or a
combination of both.
